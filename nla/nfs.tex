%%%%%%%%%%%%%%%%%%%%%%%%%%%%%%%%%%%%%%%%%%%%%%%%%%%%%%%%%%%%%%%%%%%%%%%%
%
%  This is the template file for the 6th International conference
%  NONLINEAR ANALYSIS AND EXTREMAL PROBLEMS
%  June 25-30, 2018
%  Irkutsk, Russia
%
%%%%%%%%%%%%%%%%%%%%%%%%%%%%%%%%%%%%%%%%%%%%%%%%%%%%%%%%%%%%%%%%%%%%%%%%

%  Верстка статьи осуществляется на основе стандартного класса llncs
%  (Lecture Notes in Computer Sciences), который корректируется стилевым
%  файлом конференции.
%
%  Скомпилировать файл в PDF можно двумя способами:
%  1. Использовать pdfLaTeX (pdflatex), (LaTeX+DVIPS не работает);
%  2. Использовать LuaLaTeX (XeLaTeX будет работать тоже).
%  При использовании LuaLaTeX потребуются TTF- или OTF-шрифты CMU
%  (Computer Modern Unicode). Шрифты устанавливаются либо пакетом
%  дистрибутива LaTeX cm-unicode
%              (https://www.ctan.org/tex-archive/fonts/cm-unicode),
%  либо загрузкой и установкой в операционной системе, адрес страницы:
%              http://canopus.iacp.dvo.ru/%7Epanov/cm-unicode/
%  Второй вариант не будет работать в XeLaTeX.
%
%  В MiKTeX (дистрибутив LaTeX для ОС Windows):
%  1. Пакет cm-unicode устанавливается вручную в программе MiKTeX Console.
%  2. Для верстки данного примера, а именно, картинки-заглушки необходимо,
%     также вручную, загрузить пакет pgf. Этот пакет используется популярным
%     пакетом tikz.
%  3. Тест показал, что остальные пакеты MiKTeX грузит автоматически (если
%     ему разрешено автоматически грузить пакеты). Режим автозагрузки
%     настраивается в разделе Settings в MiKTeX Console.
%
%
%  Самый простой способ сверстать статью - использовать pdfLaTeX, но
%  окончательный вариант верстки сборника будет собран в LuaLaTeX,
%  так как результат получится лучшего качества, благодаря пакету microtype и
%  использованию векторных шрифтов OTF вместо растровых pdfLaTeX.
%
%  В случае возникновения вопросов и проблем с версткой статьи,
%  пишите письма на электронную почту: eugeneai@irnok.net, Черкашин Евгений.
%
%  Новые варианты корректирующего стиля будут доступны на сайте:
%        https://github.com/eugeneai/nla-style
%        файл - nla.sty
%
%  Дальнейшие инструкции - в тексте данного шаблона. Он одновременно
%  является примером статьи.
%
%  Формат LaTeX2e!

\documentclass[12pt]{llncs}  % Необходимо использовать шрифт 12 пунктов.
% При использовании pdfLaTeX добавляется стандартный набор русификации babel.
% Если верстка производится в LuaLaTeX, то следующие три строки надо
% закомментировать, русификация будет произведена в корректирующем стиле автоматом.
\usepackage{iftex}
\ifPDFTeX
\usepackage[T2A]{fontenc}
\usepackage[utf8]{inputenc} % Кодировка utf-8, cp1251 и т.д.
\usepackage[english,russian]{babel}
\fi

\usepackage[russian]{nla}

% Для верстки в LuaLaTeX текст готовится строго в utf-8!

% В операционной системе Windows для редактирования в кодировке utf-8
% можно использовать программы notepad++ https://notepad-plus-plus.org/,
% techniccenter http://www.texniccenter.org/,
% SciTE (самая маленькая по объему программа) http://www.scintilla.org/SciTEDownload.html
% Подойдет также и встроенный в свежий дистрибутив MiKTeX редактор
% TeXworks.

% Добавляется корректирующий стилевой файл строго после babel, если он был включен.
% В параметре необходимо указать russian, что установит не совсем стандартные названия
% разделов текста, настроит переносы для русского языка как основного и т.п.


% Многие популярные пакеты (amsXXX, graphicx и т.д.) уже импортированы в корректирующий стиль.
% Если возникнут конфликты с вашими пакетами, попробуйте их отключить и сверстать
% текст как есть.
%
%


% Было б удобно при верстке сборника, чтобы названия рисунков разных авторов не пересекались.
% Чтоб минимизировать такое пересечение, рисунки можно поместить в отдельную подпапку с
% названием - фамилией автора или названием статьи.
%
% \graphicspath{{ivanov-petrov-pics/}} % Указание папки с изображениями в форматах png, pdf.
% или
% \graphicspath{{great-problem-solving-paper-pics/}}.


\begin{document}

% Текст оформляется в соответствии с классом article, используя дополнения
% AMS.
%

\title{Моделирование процессов анализа ампликонов\thanks{Работа поддержана грамнтом Иркутского научного центра СО РАН, тема\textnumero~4.2.}}
% Первый автор
\author{Е.А.Черкашин\inst{1,2,3,4}  % \inst ставит циферку над автором.
  \and  % разделяет авторов, в тексте выглядит как запятая.
% Второй автор
  А.О.Шигаров\inst{1,2} \and
  Ф.С.Малков\inst{1,4} \and
  Ю.П.Галачьянс\inst{3} \and
  А.А.Морозов\inst{3} \and
  И.С.Михайлов\inst{3} \and
  С.А.Горский\inst{2}
} % обязательное поле

% Аффилиации пишутся в следующей форме, соединяя каждый институт при помощи \and.
\institute{Иркутский научный центр СО РАН, Иркутск, Россия
  %\email{email}
  \and   % Разделяет институты и присваивает им номера по порядку.
  Институт динамики систем и теории управления им. В.М.Матросова СО РАН, Иркутск, Россия
  %\email{email}
  \and Лимнологический институт СО РАН, Иркутск, Россия
  \and Научный исследовательский иркутский государственный технический университет, Иркутск, Россия
  % \and Институт математики, экономики и информатики ИГУ, Иркутск, Россия.\\
  \email{{eugeneai,shig}@icc.ru}
}

\maketitle

\begin{abstract}
Аннотация на русском языке.

\keywords{секвенирование нового поколения, большие данные, программирование потоков данных} % в конце списка точка не ставится
\end{abstract}

В последнее десятилетие в результате изобретения методов секвенирования нового поколения (NGS, Next Generation Sequencing) и внедрение их в практику исследований биологических систем возник новый раздел молекулярной генетики -- метагеномика. Его основной объект исследования -- не отдельные микроскопические культивируемые организмы, а их сообщества (микробиомы), в пробе выделяется суммарная ДНК, секвенирование которой дает представление  о микробиоме в целом.  Метод позволяет описать значительное количество новых групп на всех таксономических уровнях.

Существуют два основных вида метагеномных исследований. Первый, более простой, называется анализом ампликонов. При этом амплифицируется и секвенируется определённый таксономический маркер, универсальный для исследуемых видов. Как правило, в роли такого маркера выступает последовательность гена малой субъединицы рибосомной РНК, этот ген широко используется в филогенетике и для него доступны многочисленные референсные последовательности.  Вторая разновидность, известная как метагеномика методом «дробовика» (shotgun metagenomics), основывается на секвенировании всей имеющейся в пробе ДНК, а не конкретных локусов. При достаточном покрытии такой подход позволяет описать не только таксономический состав сообщества, но и присутствующие в геномах представителей сообщества гены функциональных или структурных белков, в том числе – вирусных (Paez-Espino et al. 2016). На основе метагеномных данных можно устанавливать метаболические взаимодействия  в отдельных микробиомах, опираясь на существующие базы данных ePGDBs (environmental pathway/genome databases) (Hanson et al 2014).

В последние годы ампликонный анализ нашёл себе применение и в исследованиях микробиоты различных сред в озере Байкал. Описаны подлёдные бактериальные сообщества, связанные с цветением диатомей (Bashenkhaeva et al. 2015) и бактерии фотического слоя в весенний период (Михайлов и др. 2015). Исследованы бактерии, обитающие на байкальских губках (Калюжная, Кривич, Ицкович 2012, Гладких и др. 2014). Наконец, изучены бактериальные сообщества донных осадков в районах выхода углеводородов (Bukin et al. 2016, Zemskaya et al. 2015).

Для обеспечения  метагеномных исследований требуются значительные вычислительные ресурсы, а также участие квалифицированного биоинформатика в обработке и интерпретации данных. Используемое программное обеспечение анализа ампликонов включает в себя различные библиотеки модулей обработки последовательностей, например, MOTHUR (Schloss et al. 2009), USEARCH (Edgar 2010), статистические пакеты и среды разработки алгоритмов многомерного статистического анализа данных, например, R (https://www.r-project.org). Для проведения исследований с использованием обработки и анализа метагеномных данных специалисту требуются навыки составления сценариев в командной оболочке операционной системы (Linux, Windows), запуска пакетов в распределенной вычислительной среде и кластерных вычислительных системах, а также программирования на языках общего назначения, как правило, R или Python.

Целью данного исследования является разработка математического и программного обеспечения для поддержки процессов анализа результатов секвенирования нового поколения.  В


% % Рисунки и таблицы оформляются по стандарту класса article. Например,

% \begin{figure}[htb]
%   \centering
%   % Поддерживаются два формата:
%   %\includegraphics[width=0.7\linewidth]{figure.pdf} % Растровый формат
%   %\includegraphics[width=0.7\linewidth]{figure.png} % Векторный и растровый формат
%   %
%   % Векторные рисунки можно рисовать в редакторе Inkscape
%   % https://inkscape.org/ru/download/
%   % Основной формат этого редактора - SVG, поэтому рисунки необходимо экспортировать в
%   % PDF или PNG (с разрешением - минимум 150 dpi, максимум - 300dpi).
%   \begin{center}

%   \end{center}
%   \caption{Заголовок рисунка}\label{fig:example}
% \end{figure}

% Современные издательства требуют использовать кавычки-елочки << >>.

% В конце текста можно выразить благодарности, если этого не было
% сделано в ссылке с заголовка статьи, например,
% Работа выполнена при поддержке РФФИ (РНФ, другие фонды), проект \textnumero~00-00-00000.
%

% Список литературы оформляется подобно ГОСТ-2008.
% Примеры оформления находятся по этому адресу -
%     https://narfu.ru/agtu/www.agtu.ru/fad08f5ab5ca9486942a52596ba6582elit.html
%

\begin{thebibliography}{9} % или {99}, если ссылок больше десяти.
\bibitem{Gantmakher} Гантмахер~Ф.Р. Теория матриц. М.:~Наука,~1966.

\bibitem{Kholl} Современные численные методы решения обыкновенных дифференциальных уравнений~/ Под~ред.~Дж.~Холл, Дж.~Уатт. М.:~Мир,~1979.

\bibitem{Aleksandrov1} Александров~А.Ю. Об устойчивости сложных систем в критических случаях~// Автоматика и телемеханика. 2001. \textnumero~9. С.~3--13.

\bibitem{Moreau1977} Moreau~J.-J. Evolution problem associated with a moving convex set in a Hilbert space~// J.~Differential~Eq. 1977. Vol.~26. Pp.~347--374.

\bibitem{Semenov} Семенов~А.А. Замечание о вычислительной сложности известных предположительно односторонних функций~// Тр.~XII Байкальской междунар. конф. <<Методы оптимизации и их приложения>>. Иркутск, 2001. С.~142--146.

\end{thebibliography}

% После библиографического списка в русскоязычных статьях необходимо оформить
% англоязычный заголовок.

\begin{englishtitle} % Настраивает LaTeX на использование английского языка
% Этот титульный лист верстается аналогично.
\title{Modeling  processes of amplicon analysis}
% Первый автор
\author{E.Cherkashin\inst{1,2,3,4}  % \inst ставит циферку над автором.
  \and  % разделяет авторов, в тексте выглядит как запятая.
% Второй автор
  A.Shigarov\inst{1,2} \and
  F.Malkov\inst{1,4} \and
  Yu.Galachyants\inst{3} \and
  A.Morozov\inst{3} \and
  I.Mikhailov\inst{3} \and
  S.Gorsky\inst{2}
} % обязательное поле

% Аффилиации пишутся в следующей форме, соединяя каждый институт при помощи \and.
\institute{Irkutsk scientific center SB RAS, Irkutsk, Russia
  %\email{email}
  \and   % Разделяет институты и присваивает им номера по порядку.
  V.M Matrosov's Institute of system dynamics and control theory SB RAS, Irkutsk, Russia
  %\email{email}
  \and Limnological institute SB RAS, Irkutsk, Russia
  \and National research Irkutsk state technical university, Irkutsk, Russia
  % \and Институт математики, экономики и информатики ИГУ, Иркутск, Россия.\\
  \email{{eugeneai,shig}@icc.ru}
}

\maketitle
\begin{abstract}
Insert your english abstract here. Include 3-6 keywords below.

\keywords{keyword, another keyword, \ldots{}} % в конце списка точка не ставится
\end{abstract}

\end{englishtitle}


\end{document}

%%% Local Variables:
%%% mode: latex
%%% TeX-master: t
%%% End:

\documentclass[12pt]{llncs}

\usepackage{iftex}
\ifPDFTeX
\usepackage[T2A]{fontenc}
\usepackage[utf8]{inputenc} % Кодировка utf-8, cp1251 и т.д.
\usepackage[english,russian]{babel}
\fi

\usepackage[russian]{nla}

\begin{document}

\title{Моделирование процессов анализа ампликонов\thanks{Работа поддержана грантом Иркутского научного центра СО РАН, тема \textnumero~4.2.}}

\author{Е.~А.~Черкашин\inst{1,2,3,4}
  \and
  А.~О.~Шигаров\inst{1,2} \and
  Ф.~С.~Малков\inst{1,4} \and
  Ю.~П.~Галачьянс\inst{3} \and
  А.~А.~Морозов\inst{3} \and
  И.~С.~Михайлов\inst{3} \and
  С.~А.~Горский\inst{2}
}

\institute{Иркутский научный центр СО РАН, Иркутск, Россия
  \and
  Институт динамики систем и теории управления им.~В.~М.~Матросова СО РАН, Иркутск, Россия
  \and Лимнологический институт СО РАН, Иркутск, Россия
  \and Научный исследовательский иркутский государственный технический университет, Иркутск, Россия\\
  \email{\{eugeneai,shig\}@icc.ru}
}

\maketitle

\begin{abstract}
  Рассматривается задача разработки программных технологий, автоматизирующих процессы анализа данных секвенирования проб микробиома озера Байкал.  Для решения задачи разрабатываются визуальная модель представления процесса анализа, модель представления структуры информационно"=вычислительных ресурсов, а также алгоритмы синтеза плана вычислений.

\keywords{секвенирование нового поколения, большие данные, программирование потоков данных}
\end{abstract}

В последнее десятилетие в результате изобретения методов секвенирования нового поколения и внедрения их в практику исследований биологических систем, возник новый раздел молекулярной генетики -- \emph{метагеномика}. Его основной объект исследования -- не отдельные микроскопические культивируемые организмы, а их сообщества (\emph{микробиомы}).  В пробе выделяется суммарная ДНК, секвенирование которой дает представление  о микробиоме в целом.  Метод позволяет описать значительное количество новых групп микроорганизмов на всех таксономических уровнях.

Одним из видов метагеномных исследований является анализ ампликонов.  Ампликонный анализ нашёл применение в исследованиях микробиоты различных сред в озере Байкал \cite{underice}.  Для выполнения анализа требуются значительные вычислительные ресурсы, а также участие квалифицированного биоинформатика в обработке и интерпретации данных.  Исследователь строит вычислительный процесс комбинируя различные биоинформатические программные модули, модули преобразования данных, анализа данных, а также модули визуализации.  Таким образом, для проведения исследований с использованием обработки и анализа метагеномных данных специалисту-предметнику требуются навыки составления сценариев в командной оболочке операционной системы (\textsc{Linux}, \textsc{windows}), запуска пакетов в распределенной вычислительной среде и кластерных вычислительных системах, а также программирования на языках общего назначения, как правило, \textsc{r} или \textsc{python}.

Целью данного исследования является разработка математического и программного обеспечения для поддержки процессов анализа результатов секвенирования нового поколения.  Необходимо разработать и реализовать программно методики визуального моделирования вычислительного процесса анализа ампликонов таким образом, чтобы специалист-предметник самостоятельно мог создавать вычислительные цепочки, которые, затем, реализуются на распределенных гетерогенных вычислительных ресурсах.  Разрабатываемая программная реализация представляет собой информационно"=вычислительную инфраструктуру, построенную на моделях представления вычислительного процесса в виде набора операций, структуры и функций вычислительных ресурсов, а также алгоритмы планирования реализации вычислительного процесса на данных вычислительных ресурсах.

Процесс анализа ампликонов состоит из индивидуальных операций над генетическими данными.  Визуальное представление процесса осуществляется при помощи программы \emph{потоков данных} \cite{dataflow} \textsc{rapidminer studio}, где каждая операция -- это вызов модуля программ \textsc{mothur}, \textsc{r}, \textsc{python} и т.п.  Информация, передаваемая между модулями, сохраняется в файлах или в облачном хранилище.  Информационно--вычислительная инфраструктура сети иркутского научного--образовательного комплекса обладает гетерогенной структурой, куда входят как индивидуальные рабочие станции сотрудников, так и вычислительные кластеры, системы хранения данных и виртуализации.  Все элементы этой инфраструктуры потенциально могут быть задействованы для решения задач проекта.

Планирование вычислительных процессов в гетерогенных средах -- одна из сложных задач проектов анализа больших данных.  Основными решаемыми проблемами является оптимизация использования кластерных вычислительных ресурсов, планирование параллельных вычислительных схем с учетом как структуры процесса анализа ампликонов, так и свойства алгоритмов, реализующих конкретные операции; реализация облачного хранения данных с обеспечением целенаправленного доступа к конкретным данным, реализация сохранения состояния вычислительного процесса (контрольных точек).  Решение большинства указанных задач в той или иной мере реализованы в программном пакете \textsc{Orlando}, разрабатываемом в ИДСТУ СО РАН.


\begin{thebibliography}{9}
\bibitem{underice} Bashenkhaeva M.~V., Zakharova Y.~R., Petrova D.~P. Sub-Ice microalgal and bacterial communitie in freshwater Lake Baikal, Russia // Environmental Microbiology. — Vol. 70, No. 3. — P.~751-765.

\bibitem{dataflow} Johnston W.~M., Hanna J.~R.~P., Millar R.~J. Advances in Dataflow Programming Languages // ACM Computing Surveys. – 2004. – Vol. 36. – P.~1–34.

%\bibitem{mothur} Schloss P.~D. et al. Introducing mothur: open-source, platform-independent, community-supported software for describing and comparing microbial communities // Applied and Environmental Microbiology. — 2009. — Vol. 75 (No. 23). — P.~7537-7541.

\end{thebibliography}

\begin{englishtitle}

\title{Modeling  processes of amplicon analyses}

\author{E.~Cherkashin\inst{1,2,3,4}
  \and
  A.~Shigarov\inst{1,2} \and
  F.~Malkov\inst{1,4} \and
  Yu.Galachyants\inst{3} \and
  A.~Morozov\inst{3} \and\\
  I.~Mikhailov\inst{3} \and
  S.~Gorsky\inst{2}
}


\institute{Irkutsk scientific center SB RAS, Irkutsk, Russia
  \and
  V.~M.~Matrosov's Institute of system dynamics and control theory SB RAS, Irkutsk, Russia
  \and Limnological institute SB RAS, Irkutsk, Russia
  \and National research Irkutsk state technical university, Irkutsk, Russia\\
  \email{\{eugeneai,shig\}@icc.ru}
}

\maketitle
\begin{abstract}
Development of program technologies automatizing processes of analyses of data obtained from probe sequencing of lake Baikal is considered.  A visual representation model of the analysis process representation is being developed, as well as model of information-computational resources structure representation, and algorithms of plan synthesis of computations.

\keywords{new generation sequencing, big data, dataflow algorithms}
\end{abstract}
\end{englishtitle}
\end{document}

%%% Local Variables:
%%% mode: latex
%%% TeX-master: t
%%% End:

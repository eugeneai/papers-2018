\documentclass[12pt]{llncs}

\usepackage{iftex}
\ifPDFTeX
\usepackage[T2A]{fontenc}
\usepackage[utf8]{inputenc} % Кодировка utf-8, cp1251 и т.д.
\usepackage[english,russian]{babel}
\fi

\usepackage[russian]{lyap}

\begin{document}

\title{СТРУКТУРИРОВАНИЕ ЗНАНИЙ ПРОЦЕДУРЫ ТРАНСФОРМАЦИИ МОДЕЛЕЙ В МДА\thanks{Работы выполнены при поддержке Интеграционной программы ИНЦ СО РАН \textnumero~АААА-А17-117041250054-8 «Фундаментальные исследования и прорывные технологии как основа  опережающего развития Байкальского региона и его межрегиональных связей», проект \textnumero~4.2 «Применение методов NGS-BD (NextGenerationSequencing – BigData) для решения вопросов экологии».}}

\author{Е.~А.~Черкашин}

\institute{%
  Институт динамики систем и теории управления им.~В.~М.~Матросова СО РАН,
  \\
  Иркутский научный центр СО РАН,
  \\
  Научный исследовательский иркутский государственный технический университет,\\
  %\and
  Институт математики, экономики и информатики ИГУ\\
  \email{eugeneai@icc.ru}
}

\maketitle
Структурирование исходного кода программ позволяет развивать ее в различных направлениях (делить на подзадачи), обеспечивая относительную независимость между ними.  В популярных системах программирования структурирование проявляется в возможности задания структур данных (классов), процедур, функций в языке программирования, модулей, а также пакетов (packages), сервисов регистрации и загрузки пакетов, позволяющих по запросу или согласно конфигурации установить необходимые пакеты и их зависимости (dependencies).

В системах программирования Prolog, ввиду их небольшой популярности, структурирование развито не так сильно, в частности, в стандарте ISO Prolog существует возможность разбивать программу на модули, а также общая схема организации сервисов пакетов. В \cite{logtalk} предложен подход к программированию в ISO Prolog в объектной парадигме: язык программирования LogTalk.  LogTalk -- это макропакет, транслирующий исходный код в Prolog.

В \cite{icist} развит предложенный ранее подход применения логического программирования в качестве языка реализации процедур трансформации моделей (знаний о трансформации) в рамках <<Модельно-управляемой архитектуры>> (MDA).  Рассмотрена реализация трансформации в виде программы на языке LogTalk.  Трансформация представляется как сценарий синтеза платформозависимой модели (PSM) программного обеспечения (ПО).  Сценарий представляет собой объект -- инкапсуляцию процедур трансформации, посылающий запросы объектам, инкапсулирующим данные о платформонезависимой модели (PIM), и управляющий объектами, синтезирующими структуру PSM.  PIM представлена в виде RDF-графа, а PIM -- древовидная система блоков кода (объектов и литералов).

При реализации процедур трансформации использованы большинство предоставляемых LogTalk синтаксических структур и вариантов объектных иерархий.  Особенно оказались полезными параметризованные объекты и их иерархии прототипов.  Такие объекты позволяют задавать контекст выполнения метода без создания экземпляра класса.  Второй необычной но полезной структурой являются категории -- инкапсуляция реализаций логически связанных друг с другом методов -- и композиционное программирование. Все статические объекты компилятор LogTalk (например, параметризованные объекты) преобразует непосредственно в Prolog, при этом накладные расходы на производительность кода не больше 1\%.

Вышеизложенная методика программирования успешно использована в реализации процедуры синтеза исходного кода Java-модулей визуального представления структуры методо-ориентированного прикладного пакета Mothur. Основные преимущества использования логического программирования в MDA в сравнении с языками ATL -- это б\'ольшая распространенность языка, его простая и регулярная структура, возможность использовать любые структуры для представления исходной PIM, а также внешние библиотеки.

%   Zope page template

\begin{thebibliography}{9}
\bibitem{logtalk} Moura, P. A Portable and Efficient Implementation of Coinductive Logic Programming // Proceedings of the Fifteenth International Symposium on Practical Aspects of Declarative Languages (PADL 2013), 2013. Springer LNCS 7752.
\bibitem{icist} Cherkashin, E., Paramonov, V. et~al. Model driven architecture implementation using linked data // Procs. of 24th Internal Conference, ICIST 2018, Vilnus, Lithuania, October 4-6, 2018, p.~412--423.
\end{thebibliography}
\end{document}


%%% Local Variables:
%%% mode: latex
%%% TeX-master: t
%%% End:

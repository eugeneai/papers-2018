\documentclass[12pt]{llncs}

\usepackage{iftex}
\ifPDFTeX
\usepackage[T2A]{fontenc}
\usepackage[utf8]{inputenc} % Кодировка utf-8, cp1251 и т.д.
\usepackage[english,russian]{babel}
\fi

\usepackage[russian]{lyap}

\begin{document}

\title{НАЗВАНИЕ СТАТЬИ ПРОПИСНЫМИ БУКВАМИ\thanks{Работы выполнены при поддержке\ldots, проект \textnumero~\ldots.}}

\author{П.~Е.~Автор %, В.~Т.~Автор\inst{1,2} % Если нужны ссылки на институты
}

\institute{%
  Первый институт СО РАН,
  \\ % \and % Если нужна ссылка на институт.
  Второй институт СО РАН,
  \\
  \email{author@example.com} % Гиперссылка на электронный адрес
}

\maketitle

% Аннотация не заполняется

Методы решения задач оптимального управления  \ldots

\begin{equation}
 \int\limits_a^b f(x)\,\mathrm{d} x  \label{l1}
\end{equation}

Формула (\ref{l1}) \ldots

\begin{thebibliography}{9}
\bibitem{b1} Гантмахер Ф.~Р. Теория матриц.  М.:~Наука, 1966.
\bibitem{b2} Современные численные методы решения обыкновенных дифференциальных уравнений / Под ред.  Дж.~Холл, Дж.~Уатт.  М.:~Мир, 1979.
\bibitem{b3} Александров А.~Ю. Об устойчивости сложных систем в критических случаях // Автоматика и телемеханика. 2001.  No 9.  С. 3--13.
\bibitem{b4} Стрекаловский А.~С.  Об экстремальных задачах с d.c. ограничениями // Журнал вы - числительной математики и математической физики.  2001.  Т. 41, No 12.  С. 1808--1818.
\bibitem{b5}  Семенов А.~А.  Замечание о вычислительной сложности известных предположительно односторонних функций // Тр. XII Байкальской междунар.  конф. <<Методы оптимизации и их приложения>>.  Иркутск, 2001.  С. 142--146.
\end{thebibliography}
\end{document}


%%% Local Variables:
%%% mode: latex
%%% TeX-master: t
%%% End:

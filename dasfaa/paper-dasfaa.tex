\documentclass[12pt]{llncs}

\usepackage{iftex}
\ifPDFTeX
\usepackage[T2A]{fontenc}
\usepackage[utf8]{inputenc} % Кодировка utf-8, cp1251 и т.д.
\usepackage[russian,english]{babel}
\usepackage{times}
\else
\usepackage{luatextra}
  \usepackage{polyglossia}
  \setmainlanguage{russian}
  \setotherlanguage{english}
  \defaultfontfeatures{Ligatures=TeX,Numbers=Lining,Scale=MatchLowercase}

  \setmainfont{Times New Roman}
  \setsansfont{Arial}
  \setmonofont{Courier New}

  %\setmathfont{Cambria Math}%

  \newfontfamily{\cyrillicfont}{Times New Roman}
  \newfontfamily{\cyrillicfontrm}{Times New Roman}
  \newfontfamily{\cyrillicfonttt}{Courier New}
  \newfontfamily{\cyrillicfontsf}{Arial}
  \fi
\usepackage{hyperref}
\hypersetup{
    %bookmarks=true,         % show bookmarks bar?
    unicode=true,          % non-Latin characters in Acrobat’s bookmarks
    pdftoolbar=true,        % show Acrobat’s toolbar?
    pdfmenubar=true,        % show Acrobat’s menu?
    pdffitwindow=false,     % window fit to page when opened
    pdfstartview={FitH},    % fits the width of the page to the window
    pdftitle={Procedings of LYAP'2018 conference},    % title
    pdfauthor={},     % author
    pdfsubject={math},   % subject of the document
    pdfcreator={LaTeX},   % creator of the document
    pdfproducer={LaTeX}, % producer of the document
    pdfkeywords={math}, % list of keywords
    pdfnewwindow=true,      % links in new window
    colorlinks=true,       % false: boxed links; true: colored links
    linkcolor=black, %[rgb]{0 0.4 0.1},          % color of internal links
    citecolor=black, %blue,        % color of links to bibliography
    filecolor=black,      % color of file links
    urlcolor=black % [rgb]{0.3 0.0 0.3}           % color of external links
}

%\usepackage[russian]{lyap}

\begin{document}
\selectlanguage{english}
\title{An approach to creation of linked-data based document archives\thanks{Supported by organization x.}}
%
%\titlerunning{Abbreviated paper title}
% If the paper title is too long for the running head, you can set
% an abbreviated paper title here
%
\author{Evgeny Cherkashin\inst{1}\orcidID{0000-1111-2222-3333} \and
Viacheslav Paramonov\inst{1}\orcidID{1111-2222-3333-4444} \and
Alexey Shigarov\inst{1}\orcidID{1111-2222-3333-4444} \and
Andrey Mikhailov\inst{1}\orcidID{2222--3333-4444-5555}}
%
\authorrunning{E. Cherkashin et al.}
% First names are abbreviated in the running head.
% If there are more than two authors, 'et al.' is used.
%
\institute{Matrosov Institute for System Dynamics and Control Theory of Siberian Branch, Russian Academy of Sciences, Irkutsk, Russia\\
\email{\{eugene, slv, shigarov, mikhailov\}@icc.ru}}
%\url{http://www.springer.com/gp/computer-science/lncs}

%
\maketitle              % typeset the header of the contribution
%
\begin{abstract}
An amount of documents created and circulated in the world increases day-to-day. Some of new documents may use parts of archived ones (e.g. header, names of persons and organisations, tabular data and their description, footers, etc.). In this case, creator should analyse existing documents, collect and process many data for their reuse. These processes may require a lot of time and efforts. Semantic description of documents allows simplifying them. The paper presents novel approach to creation of open document catalogues that involves LODs, documents templates, and logical inference. The approach was implemented by using a contemporary open software tool-set. We used it for forming documents for educational processes regulation in departments of Irkutsk State University. It allowed to simplify parts of archived documents reusing in preparing new ones.

\keywords{Linked Open Data, component architecture, digital archive, document authoring.}

\end{abstract}

% \emph{УДК 004.652.8+004.91}

% \emph{Cherkashin E. A.,\\
% т. PhD, senior researcher,\\
% Matrosov Institute for System Dynamics and Control Theory of Siberian
% Branch, Russian Academy of Sciences\\
% tel. +7(914)870-67-54, e-mail: eugeneai@irnok.net,\\
% Orlova I. V.,\\
% ф.-м. н., доцент,\\
% National Research Irkutsk State Technical University\\
% tel. +7(914)921-07-47 e-mail: soobshenie\_1@mail.ru\\
% }


\section{Introduction}


Technologies of Linked Open Data (LOD) \cite{b1} have been suggested by
the W3C consortium to represent the semantic information in the data
published in order to provide the possibility of its processing with
software agents (Semantic Web), as well as to link all available
information into a single semantic graph using relations and universal
resource identifiers (URIs). The descriptive capabilities of the
Semantic web technologies, services of HTML5 document serving and the
LOD technologies create infrastructural basis of document publishing
with providing logical links between documents. The document content
information is associated with the content information of another one by
means of relations (links) to relevant resources. Resources represent
both the static contents and the results of the procedures execution
publishing (converting) structural data, and generating marked-up text
content.

An important advantage of using LOD in organizations' information
environments is weakening requirements for storage of the information
being published: the document itself is a data warehouse in a formalized
way. To some extent, this fact reallocates the time spent on designing
the database structure for storage partially structured documents to the
process of solving a substantive problems: the user (developer) markups
the text data with semantic structures with an instrument.

\subsection{Related works}
\label{sec:relwks}

Digital archive system development and domain adaptation is widespread
problem in IT.  Research and development is being driven in aspects of
data storage, document representation, processing, retrieval, document
flow modeling, security, etc.  At present, digital archives are integrated
with domain automation information systems, such as CRM, ERP, research and
development support platform.

One of the projects related to semantic markup control is Semantic MediaWiki; it extends a Wiki engine used for the site content representation. The text of the document is marked up with RDFa attributes with wiki tags. Semantic annotations allow development of search engines for Wiki pages that takes into account the semantics of the marked text \cite{c6}.

Unlike Semantic MediaWiki, in OntoWiki project, similar results were obtained implementing different idea. OntoWiki is based on the primary usage of the logical description of information in a semantic network. This logical structure is edited by means of the software generated user forms for the known terms in a vocabulary. The user can change only one text property of LOD lod:content, which, in General, contains HTML texts. The HTML markup is not tied to the logical structure of the displayed object (the subject). The lod:content text is edited with OntoWiki built-in WYSIWYG editor. The OntoWiki project is aimed at support of social network technologies based on Linked Data \cite{b6}.

As was mentioned above, used format of the markup of the documents is a further development of the results of the Dokieli project located at web site \url{https://github.com/linkeddata/dokieli} \cite{b14}. The project is a WYSIWYG-editor of LOD marked HTML pages, display style sheet (CSS) and built-in subgraph (its content is not visible on the page) presented in different formats (TTL, N3, JSON-LD, TriG). The user is able to add new RDFa markup with editor tools, in this case, the function of our interest is implemented in a general form. In addition, user authentication functions are implemented according to the WebID standard with client-side JavaScript. The system extensively supports commenting (text markup) with Open Annotation (oa).

The annotation engine is integrated with gitter.im service, which shows readers' comments in the form of a dialogue. All the information is stored either on servers that support the protocol Solid (\url{https://solid.mit.edu/}), or locally in the database of a browser. The edited document can be easily moved (copied) from one Solid store to another. The resulting texts can be downloaded to the user workstation as XTHML documents. Thus, Dokieli is a system supporting distributed text content editing, which fully implements the interaction with other LOD resources, and all these functions are implemented exclusively by means of the client-side browser JavaScript.

There is also a huge class of browser document editors focused on the creation of scientific publications, see, e.g., \url{http://substance.io/}. In our study, the development of tools is aimed at automating the markup of a document based on the analysis of document changes proposed in \cite{b14}, and specifically in this work, at implementing one of the infrastructural problems connected with a creation of the documents on the platform web browser controlled with LOD.

Our research is aimed at developing digital document archives enabling
synthesis of new documents combining data stored in the archive and other
services.

\subsection{Contribution}
\label{sec:contr}

Our work presence an architecture of digital archives, which allows
developers device information system and document processing services
with the following features:
\begin{itemize}
\item load LOD marked up document, extract, store in a graph and index RDF data;
\item retrieve RDF data as triples on RDF or full-text search request;
\item combine existing LOD data and its content in new documents dynamically
  thanks to relatively simple browser based context inference machine;
\item ability to use server site inference machine (Prolog) to process RDF data upon
  request from browser's part of the system.
  % \item organizes a platform for document semantic markup,
\end{itemize}

For the purpose of implementation of the feature we had to investigate existing
vocabularies used to represent generic document data LOD markup.
%We also show
%two proof-of-concept examples of the developed service usage in combination of
%other LOD aware web services of background data support and linguistic processing.

% The aim of the study is creating tools for developing digital archives
% to design and implement a domain-specific repositories of information
% objects.

Based on the produced tool set, a proof-of-concept digital archives of
documents are
developed, which intended for generation of the documentation on study
courses. The use of LOD data formats, web browser HTML processing, and
developed technologies allows us to solve a wide class of problems of
synthesis of documents, starting from the generation of substantial
parts of the texts ending the presentation of stylistic characteristics
of texts and the integration of logical markup into a global data
retrieval services.

This work continues the research outlined in \cite{b2} in the direction of
the software tools development including services storage, processing
and access to semantic information.

\section{Implementation}
\subsection{Technologies of LOD}

The technologies of LOD are based on the representation of the published
information in the form of semantically marked-up document, usually web
page or a result of a database query as HTML content. When using HTML
for presentation of content, RDFa (Resource Description Framework in
Attributes) is the primary mark-up format being supported. The markup is
a graph (subgraph) formed by the relations (triples) of the form
\texttt{<subject, relation, object>}, applied to a tree
structure of HTML with special attributes. Most of these attributes are
ignored by the web browser and do not affect the structure and style of
the displayed document.

Separately from the visual content, standard Semantic Web formats RDF,
OWL, TTL, N3 and 4, JSON-LD, RDF/JSON are used for presentation to the
LOD. These representations are extensions of standard text, XML and JSON
formats. There are several binary formats that support compact storage
of the graphs in the file system.

The representation of a graph in workstation memory devices in most
libraries and software frameworks are focused on fast search and
filtering operations of triples according to one or more criteria, as
well as the support of standard SPARQL queries. Some storage
additionally support the interpretation of relations such as
rdf:subClassOf, rdf:subPropertOf and relations derived from the
rdfs:label. This allows one to create visualization of the contents of
the graph, for example, to build the administrator interfaces for the
repositories.

LOD developers created a number of resources to discover vocabularies
(ontologies), including a set of base-level vocabularies for describing
various aspects of the content (scos, rdfs, XMLSchema, dc, prov, oa,
schema.org, etc.), graphs representing encyclopedic data, for example,
Dbedia \cite{b3} and Wikidata \cite{b4}, the Google Knowledge Graph,
services of partial automation of semantic markup of text, such as
Dbpedia Spotlight \cite{b5}, means of the distributed execution of SPARQL
queries. Created resources allow us to develop integration engines for
data and graph distributed in Internet into a single information
infrastructure.

\subsection{Digital archive tools}

Digital archive instrumental tools supporting LOD is being designed to
provide development of publishing tools of data aggregated in an
institution in a new form allowing its integration into the information
infrastructure, e.g., in form of web applications publishing data of
processing result in a scientific research.

The digital archive tool architecture is based on components. Zope
Component Architecture (ZCA) \cite{b6} used in the Python programming
environment \cite{b7} has been chosen as a architecture basis for the
implementation of the instrumental tools. Rdflib library is used for the
internal representation the LOD graphs. Graphs are serialized into
appropriate text or binary formats when transferring data to an external
process.

The toolkit consists of the following components:

\begin{itemize}
\item
  content and metadata repository with SPARQL and full-text search; the
  engines for LOD processing, based on the logical inference;
\item
  service for analysis of the stored content enriching of the archive
  with semantic information describing the content;
\item
  the tools for development of LOD applications and their user
  interfaces.
\end{itemize}



\subsection{Storing LOD documents}

The document content is stored either in the Kyotocabinet database,
either as a file in a file system. Meta-information obtained by
processing and analyzing a document is a content annotation and is
described according to LOD standard vocabulary oa as a subgraph.

Each annotation (subject) is a tree with at least two nodes formed by
the relations oa:hasTarget referring the original content and the
oa:hasBody referring the annotation. The text content of the document,
if it has been acquired, is the document annotation as well as a
structure of type Dublin Core (dc) describing meta-information found.
The contents (bytes, text) stored in Kyotocabinet is identified by its
128-bit Murmur3 hash function that allows us to create a reference key
(hash) to the contents and backward. The hash is a content identifier in
LOD. Used hashing system allows us to simply and effectively implement
computational mapping at a sufficiently low probability of clash of hash
values.

Let us list the basic properties of used vocabularies in description of
the content.

Open Annotation (oa). The vocabulary standard is adopted in 2017 by OMG, it is
used to represent the content annotation describing other content. A
browser bookmark is an example of such content.

Friend-of-a-friend (foaf) allows one to describe information about
agents: physical and legal entities, software agents.

Provenence (prov). The prov ontology is the base of description of
information flows in documents and their relationships. The prov
vocabulary is used to associate the document and owner organization or
physical entity.

In annotation, Dublin Core (dc) represents metainformation about the
creative work: authors, format, contents, description, etc.

DBPedia resource (dbr) is the namespace of the objects (resources) of
Wikipedia. The dbr ontology is used to refer to a specific entity,
geographical names, etc.

Schema.org (schema) represents the objects that are recognized by the
crawlers of Google, Yandex, Yahoo etc.

In addition to these ontologies, we use standard ontologies (RDF, RDFs,
RDFa, XSD) and ontologies from the NEPOMUK project
(\url{https://userbase.kde.org/Nepomuk}) to describe objects stored in the
full-text indexes of digital archives.

\subsection{Component of content analysis}

Meta-information about the content is enriched with results of the
analysis by so-called extractors. The extractors analyze the content and
try to acquire information about text content, author, time of content
creation, format, etc. Extractors are implemented with the utilities
such as file, extract (libextract), tracker and recoll. All these
utilities are free software and integrated with the component as
external processes.

The results of the extractor analysis are presented as a subgraph oa,
associated with the source content and stores in a metadata repository
of LOD.

\subsection{The metadata store}

The implementation of metadata LOD graph storage is realized using
ClioPatria \cite{b8} ontology server. The ClioPatria system is implemented
in the SWI-Prolog programming environment. The project is under active
development and aims to create a LOD graph storage service, implementing
all the standards of storage and access, together with tight Prolog
interpreter integration.

The system storage tier is implemented in C programming language. The
store uses compact data structures to represent triples in RAM and on
hard disk. A special feature of the storage is the fact that all triples
are stored directly in RAM, as the service has no special assumptions
about the structure of queries, which can be issued by adjacent
software. Thus, server memory space is a critical resource of the
repository.

ClioPatria developers implemented repository accessing in several ways:
the SPARQL queries according to a standard protocol, direct queries to
the ClioPatria Prolog subsystem using Pengines Protocol \cite{b9}. For the
Pengines protocols a number of APIs has been created for object-oriented
programming languages such as JavsScript, Java, Python. Using API
classes, application developers and software agents are able to interact
with the logic inference engine of the ClioPatria server.

\subsection{The inference machine}

Availability of ClioPatria's embedded programming system of SWI-Prolog
\cite{b10} and the integration with RDFa allow one to process the
marked-up digital information stored in the archive on the base of
logical inference. Prolog allows us to make queries in SPARQL and
Datalog formats, as well as to build a knowledge system to process the
received data.

As experiment, in a very short time, we managed to integrate well-known
Turbo Prolog 2.0 example of a natural-language interface to a database
on the US geography with ClioPatria's built-in SWI-Prolog RDF support
system. As a result, we developed a prototype with similar interface to
a database of active faults of Siberia, aggregated at the Institute of
earth crust SB RAS.

% Another proof-of-concept application of the logical inference at the
% server site is the support of the development of a recommender system
% \cite{b11,b12}, based on the ontological description of the set of
% characteristics of the recommended object and the user characteristics.
% (FIXME: DID NOT IMPLEMENTED AT ALL and never be) The main problem to be
% solved in such systems is the generalization of stored data that allows
% to extend the set of derived recommendations with equivalent terms in
% the same general category, as well as the narrowing the retrieval result
% number by replacing general term with specialized ones more relevant to
% the search query.

\subsection{Full-text search}

The ClioPatria system has instruments of searching the text information
for keywords, however, it is advisable to use a more powerful system of
a full-text index, for example, Elasticsearch \cite{b13}. Implementation
and integration of this service is quite simple, since every RDF graph
is representable as JSON (JSON-LD), while JSON is the primary format for
storing the indexed information in the ElasticSearch search engine; it
is only necessary to separate a particular triples on the server, which
are responsible for the representation of the resource to the user in
the search results. Elasticsearch has the means of fuzzy comparison of
terms that allows developing search subsystems retrieving relevant
information. The main aspects of modules' function are discussed later
in the article in the application section.

\section{Application of the technologies}

The designed tools used in solving a number of applied tasks built out
of digital archive services, LOD and archive components.

The Ministry of education and science of the Russian Federation after a
series of experiments switched to the large-scale incorporation of the
Bologna process in the educational environment of the Russian
Federation. One of the tasks solved in the framework of this
introduction is the transition to competence-oriented representation of
the requirements to the pedagogical process. The reform affects all
aspects of the process, including the system of classification of
specialties (introducing programs and directions, specialization
according to skill levels (bachelor, master degrees, etc.), introduction
of applied bachelor degree, list of courses, aims and objectives of the
courses (negotiation with the competencies specified in the Federal
standard), forms of conducting classes in universities (introduction of
interactive forms of learning), the distribution of lectures and
practical sessions, etc. The existing documentation is supplemented by
new forms of mandatory documents: the Fund of Assessment Means (FAM),
annotations of courses. In addition to this, the university management
with the aim of improving the quality of educational services introduced
its own additions to the form, content and requirements to the design
documentation, in particular, to the quality of the conversion to HTML
for publication on the website of the university.

For the minimum requirements fulfillment of the management of the
universities for each course, the instructor is required to prepare at
least three documents: the working program of the course, the summary
annotation, and the FAM. This set is realized for each possible
parameters combination of a course: the university, the department,
specialty, direction (profile), programme, qualification (bachelor,
specialist, undergraduate, graduate), academic or applied version of the
qualification, form of study (full-time, part-time, evening, etc.). Each
combination is reflected in the curriculum of the university from which
data is annually or twice a year should be actualized in the work
programmes.

The last five years showed the level of maturity of decisions made by
the managers of the Ministry in terms of requirements to the
presentation of the course - four generations of the standards (1, 2, 3
and 3+) are developed -- for which it was required to resubmit documents
in the appropriate updated form. The task of developing the new
textbooks now seem more simple than before as compared to the
fulfillment the course documentation requirements. Experience shows that
most instructors are not able to cope with high-quality paperwork in the
given time constraints, thus requiring the departments to hire a
secretary, whose function is to bring documentation to the required
quality level.

The solution to this problem lies in developing a software system that
allows instructors to collect the texts of the work programmes,
abstracts and FAMs from separate parts: a list of competencies and
curriculum; contents (subjects of taught module/course) and FAM, which
are shared between different versions of documents. The title pages are
generated also from data of curriculum and templates set by the
management. Microsoft Word and Excel means, that are commonly used for
this task including built-in VBA, are clearly not enough to support
problem solution.

Consider the scheme of construction of marked-up work program in the
proposed system. The presentation format is based on the results in
\cite{b4,b15}.

\begingroup\footnotesize
\begin{verbatim}
<!DOCTYPE html PUBLIC "-//W3C//DTD XHTML 1.0 Transitional//EN"
"http://www.w3.org/TR/xhtml1/DTD/xhtml1-transitional.dtd">
<html lang="ru" xmlns=http://www.w3.org/1999/xhtml
\textbf{xmlns:taa=http://irnok.net/engine/rdfa-manipulation
} xml:lang="ru" metal:define-macro="page">
<head> <!-- Connecting stylesheets
and modules -->
</head>
<body prefix="rdf:
http://www.w3.org/1999/02/22-rdf-syntax-ns# ... foaf:
http://xmlns.com/foaf/0.1/ imei: imei.html#
course:
https://irnok.net/college/plan/01.03.02-16-1234-2461_1\%D0
\%BA_PB-SM.plm.xml.xlsx-....2.3.1.html#"
resource="\#post" typeof="schema:CreativeWork sioc:Post
prov:Entity">
<!-- The application control panel -->
<main lang="ru" resource="\#annotation"
typeof="oa:Annotation"\textbf{
} id="main-document-container"><div
property="oa:hasTarget" \textbf{
resource="\#course-work-program"></div>}
<article property="oa:hasBody"
typeof="schema:Article foaf:Document curr:WorkingProgram"
resource="\#course-work-program" id="main-document">
<div
taa:content="imei:title-page"></div>
<!--Титульный лист.. -->
<div
taa:content="imei:neg-UMK"></div>
<!--Approval sheet..-->
<section id="contents" class="break-after">
<h2 class="nocount c">Table of
Contents</h2>
<div
id="tableOfContents"></div>
</section>
<section id="course-description" resource="\#description"
\textbf{property="schema:hasPart"
typeof="schema:CreativeWork">}
<div property="schema:hasPart" resource="\#purpose"
\textbf{typeof="dc:Text cnt:ContentAsText" >}
<div property="cnt:chars"
datatype="xsd:string">
<h2 property="dc:title"
datatype="xsd:string">Aims and objectives of the discipline
(module)</h2>
<p>The aim of teaching the discipline
"Technologies of programming" is the development of the
following skills ...</p>
</div>
</div>
. . . . . . . .
<div property="schema:hasPart" typeof="dc:Text
cnt:ContentAsText" resource="\#volume">
<div property="cnt:chars"
datatype="xsd:string">
<h2 property="dc:title" datatype="xsd:string">
The volume of the discipline (module) and training activities
(divided by type of training forms)</h2>
<div
taa:content="course:time-distrib"></div>
</div>
</div>
. . . . . . . .
\end{verbatim}
\endgroup

In the example, the key structures are started from \texttt{taa:} prefix. Let us
comment the structures.

The page displayed to the user is the abstract of the document
"\#annotation", and both the annotated content and the annotation text
is the same at the stage of formation of the document -- resource
"\#course-work-program". In LOD, all resources are global. This is
achieved by substitution of the default namespace - the full URI of the
current page - at the left side of the name of the resource.

The title page and the approval sheet are inserted from the templates
page ``imei.html'' (information about the Institute of mathematics,
economics and informatics of Irkutsk state university -- math.imei.ru).
The course data and the name of the specialty is filled in the templates
from the context of the main document. All the key static templates of
courses can be placed in one page of templates.

The text is divided into sections, surrounded by tags
<div> and <span> with the
relevant RDFa structures. Analysis of the experience of the developers
LOD resources has shown that for the formation of the relationship is
sufficient to use RDFa tags property, typeof and datatype. The use of
rel and about attributes are recommended to be abandoned. This makes the
structure of semantic markup more clean by reducing the number of
entities used.

Team taa:content adds to the document a text from another HTML page,
whose address is formed by the interpretation of the attribute, for
example, taa:content="course:time-distrib" loads the text of a page
generated by the table of distribution of time spent between the types
of classes (lectures, practice, laboratory, SRS, etc.). To generate such
tables, a web server has been realized for serving web pages of these
courses. Each page represents a course/module, scheduled in the
curriculum. The page content is encoded by the structure of its URL and
rendered at the time of first request. Text templates are identifier by
the id attribute.

In edit mode, the work program's tags combination with attributes
property="cnt:chars" and datatype="xsd:string" denote conversion of
their texts to editable form. For clarity, the document without
taa:-inclusions is shown to the user, indicating these inclusions with
the special tags and styles. At the end of editing, the text is stored
in the server file system. The user can then commit the changes and sync
the text of the saved document to the server control versions (git).

In another application, we implemented the service of automation of the
generation of documents with the grammar agreement engine substituting
the phrases and suffixes in its text. Consider the example of power of
attorney form representations (in Russian).

\begin{russian}\footnotesize
\begin{verbatim}
<p>Я, <span
property="fibol:designatesSignatory bibo:owner"\\
typeof="fibol:Signatory foaf:Person dbr:Principal"\\
resource="\#principal"><span property="foaf:name"
id="signatory-name" datatype="xsd:string"
class="edit">Иванова Елена
Викторовна</span>,<span
property="adoc:hasPassport" resource="\#signatory-passport"
typeof="acrt:Certification"> <span
property="acrt:qualification"
resource="dbr:Passport">паспорт</span>\ldots{}
\end{verbatim}
\end{russian}

This example expresses the fact that the principal
(fibol:designatesSignatory) is certified (adoc:hasPassport) with a
document, a passport (acrt:qualification dbr:Passport).

In the edit mode, tags with attributes
datatype="xsd:string", class="edit" are converted into the input
fields. The values of these fields propagate further in the text of the
document. Harmonization of grammar is implemented with the an id, data-
and class attributes. Attribute class=``disp'' specifies the
substitution of a phrase identified by id, in the text of the tag. The
attributes data-m and data-f specify the variant of a word with respect
to the genus of the corresponding noun. The attribute data-case
specifies the declination (case and number) for words being filled in in
the text. The grammar transformation algorithms are implemented on the
server.

Let us continue the previous example: processing grammatical structure.

\begin{russian}\footnotesize
\begin{verbatim}
...проживающ<span class="disp" id="signatory-name" data-m="ий"
data-f="ая">___</span> по
адресу:... Подпись <span class="disp string"
id="signatory-name"
data-case="gent">___</span>
удостоверяю...
\end{verbatim}
\end{russian}

The project is available as modules of the Python programming language
at: \url{https://github.com/isu-enterprise}.

\section{Conclusion and further development}
\label{sec:conc}

The paper discusses the facilities of the use of Linked Open Data (LOD)
technologies for the solution of tasks of {creating} digital document
archives with instrumental tools for creating new documents based on the
stored data. Layout is carried out using algorithms that interpret the
relationships between the elements of the HTML tree. The layout
algorithms are triggered in the presence of corresponding conditions in
the nodes of the document tree. Each algorithm makes changes in the
tree, resulting the final content and styles of the document. Forming
documents are stored on servers that support the HTTP Protocol.

An example of use of the development tools for the document synthesis in
the educational environment and legal documents has been presented.
Further development of this project is carried in the following
directions: a) improvement of the tools of typesetting documents, b)
automation of the layout based on analysis of changes in the documents
\cite{b2}, c) implementation in practical applications, d) collecting of
additional information about user needs, d) develop a regulated access
to information stored in digital archives. The created software tools
and technologies aimed at the development of a software infrastructure
of a global electronic document flow, allowing individuals and
organizations to share documents, whose logical structure is partially
formalizable within the framework of the traditional software of
document automation.

\section{Acknowledgments}

The results are obtained with the active use of the network
infrastructure of Telecommunications of Center of collective use
"Integrated information-computing network of Irkutsk
scientific-educational complex" (\url{http://net.icc.ru}).

\begin{thebibliography}{99}
\bibitem{b1} Bizer Ch., Heath T., Berners-Lee T. Linked Data -- The Story So
  Far // International Journal on Semantic Web and Information Systems.
  2009. Vol. 5 (3). P. 1--22. \url{doi:10.4018/jswis.2009081901. }

\bibitem{c6} N.Heino, S.Tramp, N.Heino, S.Auer. Managing Web Content using Linked Data Principles – Combining semantic structure with dynamic content syndication. Computer Software and Applications Conference (COMPSAC), 2011 IEEE 35th Annual. pp. 245 - 250. URL: \url{http://svn.aksw.org/papers/2011/COMPSAC_lod2.eu/public.pdf} (access date: 30.05.2013).
\bibitem{b6}
  Baiju M. A Comprehensive Guide to Zope Component Architecture.
  [Electronic resource]
URL: \url{http://muthukadan.net/docs/zca.html} (access date - 01.07.2017)
\bibitem{b14}
  Capadisli, S., Guy, A., Verborgh, R., Lange, C., Auer, S., Berners-Lee, T.: Decentralised Authoring, Annotations and Notifications for a Read-Write Web with dokieli, Procs of ICWE international conference, 5-8 June, 2017, Rome, Italy. p. 469--481 URL: \url{https://link.springer.com/chapter/10.1007/978-3-319-60131-1_33}



\bibitem{b2}
  Cherkashin, E. A., Belykh, P. V. et al. An approach to managing a site
  content on the base of RDF technologies // Materials of All-Russian
  conference with international participation "Knowledge -- Ontology --
  Theory" (KONT-2013), 8 -- 10 October. 2013. Vol. 2. Novosibirsk.
  Publishing house "AIC Price-courier".
\bibitem{b3}
  Lehmann J., Isele R., Jakob M., Jentzsch A., Kontokostas D., et al.
  DBpedia -- A Large-scale, Multilingual Knowledge Base Extracted from
  Wikipedia // Semantic Web Journal. 2015. Vol. 6, No. 2, P. 167-195,
  IOS Press.
\bibitem{b4}
  Krötzsch M. How to use Wikidata: Things to make and do with 40 million
  statements // In Keynote at the 10th Wikimania Conference. 2014.
\bibitem{b5}
  Daiber J., Jakob M., Mendes P. Improving Efficiency and Accuracy in
  Multilingual Entity Extraction // Proceedings of the 9th International
  Conference on Semantic Systems (I-Semantics), 2013.
\bibitem{b7}
  Langtangen H. A Primer on Scientific Programming with Python (Texts in
  Computational Science and Engineering) 3rd ed. Springer. 2012. 798 P.
\bibitem{b8}
  Wielemaker J., Beek W., Hildebrand M., Ossenbruggen J. ClioPatria: A
  SWI-Prolog infrastructure for the Semantic Web //  Semantic Web. 2016.
  Vol. 7(5). P. 529-541, DOI: 10.3233/SW-150191
\bibitem{b9}
  Lager T., Wielemaker J. Pengines: Web Logic Programming Made Easy //
  Theory and Practice of Logic Programming. 2014. Vol. 14(4-5),
DOI: \url{10.1017/S1471068414000192}
\bibitem{b10}
  Wielemaker J., Schreiber G., Wielinga B. Prolog-Based Infrastructure
  for RDF: Scalability and Performance // In: Fensel D., Sycara K.,
  Mylopoulos J. (eds) The Semantic Web - ISWC 2003. ISWC 2003. Lecture
  Notes in Computer Science. 2003. Vol. 2870. Springer, Berlin,
  Heidelberg.
% \bibitem{b11}
%   Nefedova,Yu. S. The architecture of hybrid Recommender system GEFEST
%   (Generation--Expansion--Filtering--Sorting--Truncation) // Systems and
%   software of Informatics. 2012, vol. 22, issue.2, pp. 176--196.
% \bibitem{b12}
%   Beel J., Gripp B., Langer S., Breitinger C. Research-paper recommender
%   systems: a literature survey // International Journal on Digital
%   Libraries. 2016. Vol. 17. P. 305. doi:10.1007/s00799-015-0156-0.
%   (access date - 12.12.2016)
\bibitem{b13}
  Kuć R., Rogoziński M. Mastering Elasticsearch - Second Edition. Packt
  Publishing. 2015. 372 p. ISBN-9781783553792
\bibitem{b15}
  Heino N, Tramp S, Auer S, et al. Managing Web Content using Linked
  Data Principles -- Combining semantic structure with dynamic content
  syndication // Computer Software and Applications Conference
  (COMPSAC), IEEE 35th Annual. 2011. P. 245-250. [Electronic
  resource]
  URL:http://svn.aksw.org/papers/2011/COMPSAC\_lod2.eu/public.pdf
  (access date - 30.05.2013).
\end{thebibliography}
\end{document}

%%% Local Variables:
%%% mode: latex
%%% TeX-master: t
%%% End:

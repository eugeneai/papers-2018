\title{An approach to creation of linked-data based document archives\thanks{Supported by organization x.}}
%
%\titlerunning{Abbreviated paper title}
% If the paper title is too long for the running head, you can set
% an abbreviated paper title here
%
\author{Evgeny Cherkashin\inst{1}\orcidID{0000-1111-2222-3333} \and
Viacheslav Paramonov\inst{1}\orcidID{1111-2222-3333-4444} \and
Alexey Shigarov\inst{1}\orcidID{1111-2222-3333-4444} \and
Andrey Mikhailov\inst{1}\orcidID{2222--3333-4444-5555}}
%
\authorrunning{E. Cherkashin et al.}
% First names are abbreviated in the running head.
% If there are more than two authors, 'et al.' is used.
%
\institute{Matrosov Institute for System Dynamics and Control Theory of Siberian Branch, Russian Academy of Sciences, Irkutsk, Russia
\and
\email{\{eugene, slv, shigarov, mikhailov\}@icc.ru}\\
%\url{http://www.springer.com/gp/computer-science/lncs} 

%
\maketitle              % typeset the header of the contribution
%
\begin{abstract}
An amount of documents created and circulated in the world increases day-to-day. Some of new documents may use parts of archived ones (e.g. header, names of persons and organisations, tabular data and their description, footers, etc.). In this case, creator should analyse existing documents, collect and process many data for their reuse. These processes may require a lot of time and efforts. Semantic description of documents allows simplifying them. The paper presents novel approach to creation of open document catalogues that involves LODs, documents templates, and logical inference. The approach was implemented by using a contemporary open software tool-set. We used it for forming documents for educational processes regulation in departments of Irkutsk State University. It allowed to simplify parts of archived documents reusing in preparing new ones.  

\keywords{Linked Open Data \and component architecture \and digital archive \and document authoring.}
\end{abstract}
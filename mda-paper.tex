%% bare_conf.tex
%% V1.4a
%% 2014/09/17
%% by Michael Shell
%% See:
%% http://www.michaelshell.org/
%% for current contact information.
%%
%% This is a skeleton file demonstrating the use of IEEEtran.cls
%% (requires IEEEtran.cls version 1.8a or later) with an IEEE
%% conference paper.
%%
%% Support sites:
%% http://www.michaelshell.org/tex/ieeetran/
%% http://www.ctan.org/tex-archive/macros/latex/contrib/IEEEtran/
%% and
%% http://www.ieee.org/

%%*************************************************************************
%% Legal Notice:
%% This code is offered as-is without any warranty either expressed or
%% implied; without even the implied warranty of MERCHANTABILITY or
%% FITNESS FOR A PARTICULAR PURPOSE!
%% User assumes all risk.
%% In no event shall IEEE or any contributor to this code be liable for
%% any damages or losses, including, but not limited to, incidental,
%% consequential, or any other damages, resulting from the use or misuse
%% of any information contained here.
%%
%% All comments are the opinions of their respective authors and are not
%% necessarily endorsed by the IEEE.
%%
%% This work is distributed under the LaTeX Project Public License (LPPL)
%% ( http://www.latex-project.org/ ) version 1.3, and may be freely used,
%% distributed and modified. A copy of the LPPL, version 1.3, is included
%% in the base LaTeX documentation of all distributions of LaTeX released
%% 2003/12/01 or later.
%% Retain all contribution notices and credits.
%% ** Modified files should be clearly indicated as such, including  **
%% ** renaming them and changing author support contact information. **
%%
%% File list of work: IEEEtran.cls, IEEEtran_HOWTO.pdf, bare_adv.tex,
%%                    bare_conf.tex, bare_jrnl.tex, bare_conf_compsoc.tex,
%%                    bare_jrnl_compsoc.tex, bare_jrnl_transmag.tex
%%*************************************************************************


% *** Authors should verify (and, if needed, correct) their LaTeX system  ***
% *** with the testflow diagnostic prior to trusting their LaTeX platform ***
% *** with production work. IEEE's font choices and paper sizes can       ***
% *** trigger bugs that do not appear when using other class files.       ***                          ***
% The testflow support page is at:
% http://www.michaelshell.org/tex/testflow/



\documentclass[conference,a4paper]{IEEEtran}
\usepackage[T1]{fontenc}
\usepackage[utf8]{inputenc}
\usepackage{currvita}
% Some Computer Society conferences also require the compsoc mode option,
% but others use the standard conference format.
%
% If IEEEtran.cls has not been installed into the LaTeX system files,
% manually specify the path to it like:
% \documentclass[conference]{../sty/IEEEtran}





% Some very useful LaTeX packages include:
% (uncomment the ones you want to load)


% *** MISC UTILITY PACKAGES ***
%
%\usepackage{ifpdf}
% Heiko Oberdiek's ifpdf.sty is very useful if you need conditional
% compilation based on whether the output is pdf or dvi.
% usage:
% \ifpdf
%   % pdf code
% \else
%   % dvi code
% \fi
% The latest version of ifpdf.sty can be obtained from:
% http://www.ctan.org/tex-archive/macros/latex/contrib/oberdiek/
% Also, note that IEEEtran.cls V1.7 and later provides a builtin
% \ifCLASSINFOpdf conditional that works the same way.
% When switching from latex to pdflatex and vice-versa, the compiler may
% have to be run twice to clear warning/error messages.






% *** CITATION PACKAGES ***
%
\usepackage{cite}
% cite.sty was written by Donald Arseneau
% V1.6 and later of IEEEtran pre-defines the format of the cite.sty package
% \cite{} output to follow that of IEEE. Loading the cite package will
% result in citation numbers being automatically sorted and properly
% "compressed/ranged". e.g., [1], [9], [2], [7], [5], [6] without using
% cite.sty will become [1], [2], [5]--[7], [9] using cite.sty. cite.sty's
% \cite will automatically add leading space, if needed. Use cite.sty's
% noadjust option (cite.sty V3.8 and later) if you want to turn this off
% such as if a citation ever needs to be enclosed in parenthesis.
% cite.sty is already installed on most LaTeX systems. Be sure and use
% version 5.0 (2009-03-20) and later if using hyperref.sty.
% The latest version can be obtained at:
% http://www.ctan.org/tex-archive/macros/latex/contrib/cite/
% The documentation is contained in the cite.sty file itself.

\usepackage{hyperref}
\hypersetup{
    % bookmarks=true,         % show bookmarks bar?
    unicode=true,          % non-Latin characters in Acrobat’s bookmarks
    pdftoolbar=true,        % show Acrobat’s toolbar?
    pdfmenubar=true,        % show Acrobat’s menu?
    pdffitwindow=false,     % window fit to page when opened
    pdfstartview={FitH},    % fits the width of the page to the window
    %pdftitle={},    % title
    %pdfauthor={Author},     % author
    %pdfsubject={Subject},   % subject of the document
    %pdfcreator={Creator},   % creator of the document
    %pdfproducer={Producer}, % producer of the document
    %pdfkeywords={keyword1, key2, key3}, % list of keywords
    %pdfnewwindow=true,      % links in new PDF window
    colorlinks=true,       % false: boxed links; true: colored links
    linkcolor=black,          % color of internal links (change box color with linkbordercolor)
    citecolor=black,        % color of links to bibliography
    filecolor=black,      % color of file links
    urlcolor=black,           % color of external links
    final=true
  }




% *** GRAPHICS RELATED PACKAGES ***
%
\ifCLASSINFOpdf
  % \usepackage[pdftex]{graphicx}
  % declare the path(s) where your graphic files are
  % \graphicspath{{../pdf/}{../jpeg/}}
  % and their extensions so you won't have to specify these with
  % every instance of \includegraphics
  % \DeclareGraphicsExtensions{.pdf,.jpeg,.png}
\else
  % or other class option (dvipsone, dvipdf, if not using dvips). graphicx
  % will default to the driver specified in the system graphics.cfg if no
  % driver is specified.
  % \usepackage[dvips]{graphicx}
  % declare the path(s) where your graphic files are
  % \graphicspath{{../eps/}}
  % and their extensions so you won't have to specify these with
  % every instance of \includegraphics
  % \DeclareGraphicsExtensions{.eps}
\fi
% graphicx was written by David Carlisle and Sebastian Rahtz. It is
% required if you want graphics, photos, etc. graphicx.sty is already
% installed on most LaTeX systems. The latest version and documentation
% can be obtained at:
% http://www.ctan.org/tex-archive/macros/latex/required/graphics/
% Another good source of documentation is "Using Imported Graphics in
% LaTeX2e" by Keith Reckdahl which can be found at:
% http://www.ctan.org/tex-archive/info/epslatex/
%
% latex, and pdflatex in dvi mode, support graphics in encapsulated
% postscript (.eps) format. pdflatex in pdf mode supports graphics
% in .pdf, .jpeg, .png and .mps (metapost) formats. Users should ensure
% that all non-photo figures use a vector format (.eps, .pdf, .mps) and
% not a bitmapped formats (.jpeg, .png). IEEE frowns on bitmapped formats
% which can result in "jaggedy"/blurry rendering of lines and letters as
% well as large increases in file sizes.
%
% You can find documentation about the pdfTeX application at:
% http://www.tug.org/applications/pdftex





% *** MATH PACKAGES ***
%
%\usepackage[cmex10]{amsmath}
% A popular package from the American Mathematical Society that provides
% many useful and powerful commands for dealing with mathematics. If using
% it, be sure to load this package with the cmex10 option to ensure that
% only type 1 fonts will utilized at all point sizes. Without this option,
% it is possible that some math symbols, particularly those within
% footnotes, will be rendered in bitmap form which will result in a
% document that can not be IEEE Xplore compliant!
%
% Also, note that the amsmath package sets \interdisplaylinepenalty to 10000
% thus preventing page breaks from occurring within multiline equations. Use:
%\interdisplaylinepenalty=2500
% after loading amsmath to restore such page breaks as IEEEtran.cls normally
% does. amsmath.sty is already installed on most LaTeX systems. The latest
% version and documentation can be obtained at:
% http://www.ctan.org/tex-archive/macros/latex/required/amslatex/math/





% *** SPECIALIZED LIST PACKAGES ***
%
%\usepackage{algorithmic}
% algorithmic.sty was written by Peter Williams and Rogerio Brito.
% This package provides an algorithmic environment fo describing algorithms.
% You can use the algorithmic environment in-text or within a figure
% environment to provide for a floating algorithm. Do NOT use the algorithm
% floating environment provided by algorithm.sty (by the same authors) or
% algorithm2e.sty (by Christophe Fiorio) as IEEE does not use dedicated
% algorithm float types and packages that provide these will not provide
% correct IEEE style captions. The latest version and documentation of
% algorithmic.sty can be obtained at:
% http://www.ctan.org/tex-archive/macros/latex/contrib/algorithms/
% There is also a support site at:
% http://algorithms.berlios.de/index.html
% Also of interest may be the (relatively newer and more customizable)
% algorithmicx.sty package by Szasz Janos:
% http://www.ctan.org/tex-archive/macros/latex/contrib/algorithmicx/




% *** ALIGNMENT PACKAGES ***
%
%\usepackage{array}
% Frank Mittelbach's and David Carlisle's array.sty patches and improves
% the standard LaTeX2e array and tabular environments to provide better
% appearance and additional user controls. As the default LaTeX2e table
% generation code is lacking to the point of almost being broken with
% respect to the quality of the end results, all users are strongly
% advised to use an enhanced (at the very least that provided by array.sty)
% set of table tools. array.sty is already installed on most systems. The
% latest version and documentation can be obtained at:
% http://www.ctan.org/tex-archive/macros/latex/required/tools/


% IEEEtran contains the IEEEeqnarray family of commands that can be used to
% generate multiline equations as well as matrices, tables, etc., of high
% quality.




% *** SUBFIGURE PACKAGES ***
%\ifCLASSOPTIONcompsoc
%  \usepackage[caption=false,font=normalsize,labelfont=sf,textfont=sf]{subfig}
%\else
%  \usepackage[caption=false,font=footnotesize]{subfig}
%\fi
% subfig.sty, written by Steven Douglas Cochran, is the modern replacement
% for subfigure.sty, the latter of which is no longer maintained and is
% incompatible with some LaTeX packages including fixltx2e. However,
% subfig.sty requires and automatically loads Axel Sommerfeldt's caption.sty
% which will override IEEEtran.cls' handling of captions and this will result
% in non-IEEE style figure/table captions. To prevent this problem, be sure
% and invoke subfig.sty's "caption=false" package option (available since
% subfig.sty version 1.3, 2005/06/28) as this is will preserve IEEEtran.cls
% handling of captions.
% Note that the Computer Society format requires a larger sans serif font
% than the serif footnote size font used in traditional IEEE formatting
% and thus the need to invoke different subfig.sty package options depending
% on whether compsoc mode has been enabled.
%
% The latest version and documentation of subfig.sty can be obtained at:
% http://www.ctan.org/tex-archive/macros/latex/contrib/subfig/




% *** FLOAT PACKAGES ***
%
%\usepackage{fixltx2e}
% fixltx2e, the successor to the earlier fix2col.sty, was written by
% Frank Mittelbach and David Carlisle. This package corrects a few problems
% in the LaTeX2e kernel, the most notable of which is that in current
% LaTeX2e releases, the ordering of single and double column floats is not
% guaranteed to be preserved. Thus, an unpatched LaTeX2e can allow a
% single column figure to be placed prior to an earlier double column
% figure. The latest version and documentation can be found at:
% http://www.ctan.org/tex-archive/macros/latex/base/


%\usepackage{stfloats}
% stfloats.sty was written by Sigitas Tolusis. This package gives LaTeX2e
% the ability to do double column floats at the bottom of the page as well
% as the top. (e.g., "\begin{figure*}[!b]" is not normally possible in
% LaTeX2e). It also provides a command:
%\fnbelowfloat
% to enable the placement of footnotes below bottom floats (the standard
% LaTeX2e kernel puts them above bottom floats). This is an invasive package
% which rewrites many portions of the LaTeX2e float routines. It may not work
% with other packages that modify the LaTeX2e float routines. The latest
% version and documentation can be obtained at:
% http://www.ctan.org/tex-archive/macros/latex/contrib/sttools/
% Do not use the stfloats baselinefloat ability as IEEE does not allow
% \baselineskip to stretch. Authors submitting work to the IEEE should note
% that IEEE rarely uses double column equations and that authors should try
% to avoid such use. Do not be tempted to use the cuted.sty or midfloat.sty
% packages (also by Sigitas Tolusis) as IEEE does not format its papers in
% such ways.
% Do not attempt to use stfloats with fixltx2e as they are incompatible.
% Instead, use Morten Hogholm'a dblfloatfix which combines the features
% of both fixltx2e and stfloats:
%
% \usepackage{dblfloatfix}
% The latest version can be found at:
% http://www.ctan.org/tex-archive/macros/latex/contrib/dblfloatfix/




% *** PDF, URL AND HYPERLINK PACKAGES ***
%
%\usepackage{url}
% url.sty was written by Donald Arseneau. It provides better support for
% handling and breaking URLs. url.sty is already installed on most LaTeX
% systems. The latest version and documentation can be obtained at:
% http://www.ctan.org/tex-archive/macros/latex/contrib/url/
% Basically, \url{my_url_here}.




% *** Do not adjust lengths that control margins, column widths, etc. ***
% *** Do not use packages that alter fonts (such as pslatex).         ***
% There should be no need to do such things with IEEEtran.cls V1.6 and later.
% (Unless specifically asked to do so by the journal or conference you plan
% to submit to, of course. )


% correct bad hyphenation here
\hyphenation{op-tical net-works semi-conduc-tor}

\providecommand\url[1]{\texttt{#1}}

\begin{document}
\urlstyle{tt}

%
% paper title
% Titles are generally capitalized except for words such as a, an, and, as,
% at, but, by, for, in, nor, of, on, or, the, to and up, which are usually
% not capitalized unless they are the first or last word of the title.
% Linebreaks \\ can be used within to get better formatting as desired.
% Do not put math or special symbols in the title.
\date{}
\title{Model Driven Architecture Based on Linked Data}

%\author{Evgeny Cherkashin, Alexey Kopaigorodsky, Ljubica Kazi.}

% author names and affiliations
% use a multiple column layout for up to three different
% affiliations
% \author{\IEEEauthorblockN{Michael Shell}
% \IEEEauthorblockA{School of Electrical and\\Computer Engineering\\
% Georgia Institute of Technology\\
% Atlanta, Georgia 30332--0250\\
% Email: http://www.michaelshell.org/contact.html}
% \and
% \IEEEauthorblockN{Homer Simpson}
% \IEEEauthorblockA{Twentieth Century Fox\\
% Springfield, USA\\
% Email: homer@thesimpsons.com}
% \and
% \IEEEauthorblockN{James Kirk\\ and Montgomery Scott}
% \IEEEauthorblockA{Starfleet Academy\\
% San Francisco, California 96678--2391\\
% Telephone: (800) 555--1212\\
% Fax: (888) 555--1212}}

% conference papers do not typically use \thanks and this command
% is locked out in conference mode. If really needed, such as for
% the acknowledgment of grants, issue a \IEEEoverridecommandlockouts
% after \documentclass

% for over three affiliations, or if they all won't fit within the width
% of the page, use this alternative format:
%

\renewcommand\IEEEkeywordsname{Keywords}
\DeclareRobustCommand*{\IEEEauthorrefmark}[1]{\raisebox{0pt}[0pt][0pt]{\textsuperscript{\footnotesize #1}}}
\author{Evgeny Cherkashin\IEEEauthorrefmark{1,}\IEEEauthorrefmark{2,}\IEEEauthorrefmark{5,}\IEEEauthorrefmark{6},
Alexey Kopaygorodsky\IEEEauthorrefmark{3,}\IEEEauthorrefmark{6},
Ljubica Kazi\IEEEauthorrefmark{4},\\
Alexey Shigarov\IEEEauthorrefmark{1,}\IEEEauthorrefmark{2},
and
Vyacheslav Paramonov\IEEEauthorrefmark{2,}\IEEEauthorrefmark{5}\\[0.3em]
%
\IEEEauthorblockA{\IEEEauthorrefmark{1}Irkutsk Scientific Center of SB RAS, 134 Lermontov Street, Russia, 664033}
\IEEEauthorblockA{\IEEEauthorrefmark{2}Matrosov Institute for System Dynamics and Control Theory of SB RAS, 134 Lermontov Street, Russia, 664033}
\IEEEauthorblockA{\IEEEauthorrefmark{3}Melentiev Energy Systems Institute of Siberian Branch of the Russian Academy of Sciences,\\ 130 Lermontov Street, Russia, 664033}
\IEEEauthorblockA{\IEEEauthorrefmark{4}University of Novi Sad, Technical faculty "Mihajlo Pupin", \DJ{}ure \DJ{}akovi\'ca bb,
 Zrenjanin, Serbia, 23101}
\IEEEauthorblockA{\IEEEauthorrefmark{5}Irkutsk State University, 20 Gagarina Avenue, Russia, 664002}
\IEEEauthorblockA{\IEEEauthorrefmark{6}National Research Irkutsk State Technical University, 83 Lermontov Street, Russia, 664074}
E-mail:\texttt{eugeneai@icc.ru,digger@istu.edu,ljubica.kazi@gmail.com}}


% use for special paper notices
%\IEEEspecialpapernotice{(Invited Paper)}




% make the title area
\maketitle

% As a general rule, do not put math, special symbols or citations
% in the abstract
\begin{abstract}
We consider tools for developing information systems with extensive use of Linked Open Data technologies (LOD).  The LOD technologies allows software designers to develop program systems data integrated by means of common ontologies and web protocols.  An application of semantic models at run time  allowed software designers to partially cope with problem of requirements and data structure evolution in lifespan of information systems.  A part of system configuration (static or dynamic) is build on domain ontology model interpretation, thus modifying information system behavior, resulting in a wider flexibility of system functioning.

The present research deals with developing software instrumental tools for general design of information systems using both Model Driven Architecture (MDA) and LOD.  MDA Platform Independent Model (PIM) is expressed as set of UML diagrams (Class Diagram, State Diagram, \emph{etc.}).  The model form a namespace of a LOD graph.  All the PIM entities (MOF structures) are defined as ontology resources, \emph{i.e.} with URI references to LOD terms.  This allows us to translate PIM UML model to set of triples and store them in an ontology warehouse for further transformation into a Platform Specific Model (PSM).

The ClioPatria ontology server and the SWI Prolog language are used as tools of MDA model querying and processing.

One of the problem of tools construction of such kind is the absence of visual editors of UML supporting ontology models.  The UML 2.4 standard implicitly suggests coupling stereotypes and tag values forming a data format representing stereotype aspects of the UML entities.  Thus, one of the research direction is to construct new or adapt existing UML modeling tool to support LOD in UML diagrams.

The tools will allow us to join the MDA static means of code generation and configuration at development stage with the techniques of flexible data structure processing at run time, thus, producing even more productive information system development and maintenance.  This research corresponds to nowadays direction of Semantic Web Software Engineering approach (J.Z. Pan, Y. Zhao, 2014) with aim of construction LOD based information systems as a peculiar case of Semantic Web technologies.
\end{abstract}
\vspace{1em}
\begin{IEEEkeywords} model driven architecture; linked open data; semantic web enabled software engineering; logic programming; knowledge--based systems
\end{IEEEkeywords}

% no keywords




% For peer review papers, you can put extra information on the cover
% page as needed:
% \ifCLASSOPTIONpeerreview
% \begin{center} \bfseries EDICS Category: 3-BBND \end{center}
% \fi
%
% For peerreview papers, this IEEEtran command inserts a page break and
% creates the second title. It will be ignored for other modes.
\IEEEpeerreviewmaketitle



\section{Introduction}
% no \IEEEPARstart
We consider tools for developing information systems with extensive use of Linked Open Data technologies (LOD) \cite{Bizer}.  The LOD technologies allows software designers to develop program systems data integrated by means of common ontologies and web protocols \cite{Cherk}.  An application of semantic models at run time \cite{Kopay} allowed software designers to partially cope with problem of requirements and data structure evolution in lifespan of information systems.  A part of system configuration (static or dynamic) is build on domain ontology model interpretation, thus modifying information system behavior, resulting in a wider flexibility of system functioning.

The present research deals with developing software instrumental tools for general design of information systems using both Model Driven Architecture (MDA) and LOD \cite{MDA}.  MDA Platform Independent Model (PIM) is expressed as set of UML diagrams (Class Diagram, State Diagram, \emph{etc.}).  The model form a \verb~namespace~ of a LOD graph.  All the PIM entities (MOF structures) are defined as ontology resources, \emph{i.e.} with URI references to LOD terms.  This allows us to translate PIM UML model to set of triples and store them in an ontology warehouse for further transformation into a Platform Specific Model (PSM). The stage of designing PIM is very important as it in principle provides an evolution of the information system implementation within the development of the programming technologies in the future. %This means that correct and fully described PIM has more value versus PSM.
The \verb~ClioPatria~ ontology server and the SWI Prolog language \cite{Clio} are used as tools of MDA model querying and processing.

We distinct two popular approaches of implementing PIM to PSM transformation, taking into account the runtime features and other implementation platform features.  The first one is a library-based approach that allows us to build the resulting PSM and the source code easy enough.  Transformation functions from the library are applied to predicates or implementation of complex steps or subtasks. Usage of interpreted or ahead-of-time-compilation languages as Python, Perl, PHP, ASP.NET, JavaScript, HTML as a target language for PSMs decreases the development time and make it less complex.  Thus, application of the libraries is the direct interaction of PIM as a superposition of applied function.

The second popular approach is the multistage transformation of the source PIM (\textbf{From previous paper}).  This research uses the Logtalk language, a macro package of the Prolog logical language.  The main advantage of the language usage is knowledge structuring thanks to the object-orientation.  Objects are used for representation of the source PIM model and the resulting PIM structures as well as source generators.  Moreover, the transformation objects is easily integrated with the libraries processing LOD and LOD warehouses.

\textbf{The ontological approach may be used not only to describe the model and data, but also to describe the interface elements, their behavior, the rights and limitations of users and others aspects of specific realisation.}

One of the problem of tools construction of such kind is the absence of visual editors of UML supporting ontology models.  The UML 2.4 standard implicitly suggests coupling stereotypes and tag values forming a data format representing stereotype aspects of the UML entities.  Thus, one of the research direction is to construct new or adapt existing UML modeling tool to support LOD in UML diagrams.

Linked Open Data (LOD) [1] technology has been suggested by W3C consortium to represent the semantic information in the published web content in a way that provides not only the possibility of its processing with software agents (Semantic Web), but also to link all available information into a single semantic graph using relations and global universal identifiers (URIs) of resources. The descriptive capabilities of semantic web technologies, HTML5 document publishing tools, and LOD technologies form an infrastructural basis of authoring and publishing documents. The document is constructed out of individual parts (text and images) that are loaded from other servers with links to relevant resources. The resources represent both the static content and text content as a result of a data conversion algorithm execution. The LOD provide a logical markup for the information presented in a document, informative basis for the different variants of visual representation and interpretation, logical connections with other documents, export information into other documents, procedural processing, etc. An important advantage of LOD usage in information environments is a weakening of the requirements to the information warehouses: the document itself is a formalized data warehouse. In some extent, this allows reallocation of the time spent on designing the database structure for storage of partially formalized documents to the process of solving a domain problem: the user (developer) markups the document text data with semantic meaning.

\section{Transformation software architecture}
\label{sec:arch}

The architecture of the system is presented in Fig. 1 (all relations shown in the figure are bidirectional). The document is downloaded from a server where it is composed from a content stored in a database or a file system, results of an algorithmic data conversion of other sources, and elements of a template that includes the user view interface elements and interpretation modules on the displayed page. In the formation of the final form of the document, interpretation modules request the required data and resources on the source server or other Internet servers.

\begin{figure}[t]
  \centering
  \caption{The general architecture of the PIM transformation system}
  \label{fig:arch}
\end{figure}


Interpretation modules of the semantic markup are implemented with client-side JavaScript. They are executed as soon as the main content is loaded by the browser. The modules scan the tree structure of the document, recognizing the conditions of their activation. If the scan was successful, the body of the module is executed, changing of the document content (tree structure). The document composition finalizes as soon as all conditions were met and all commands activated.

A database of the system is a set of tools for storing and serving multi-format data. The body of the document loaded at the first stage is stored in XML (XHTML) files in the file system on the server. The file system is wrapped with a document versions tracking layer to monitor the document dynamics and create integrated backup copies. This level is quite easy to be implemented using modern version control systems like GIT or Subversion.

In order to organize standard SPARQL data source access and full-text search, storage components for the logical layer and related data as a graph of triples have been developing. ClioPatria [4] and Jena are the component implementations. The first system is interesting because it is implemented entirely with the programming languages Prolog and C. It supports several formats of compact storage of data, and tight integration of triple data with the runtime and language environment of Prolog. A regulated access to the triples from JavaScript is supported using special Pengines protocol [5]. There are a number of implementation libraries for the protocols for popular programming languages, including client-side JavaScript, Java, Python. As an alternative, Java-based Jena library can be used for the service realization.

A full-text index is provided by Elasticsearch indexing engine. The implementation of the index service is simple enough because any RDF graph is representable as JSON (JSON-LD) document, JSON is the main format of storage of indexed information in ElasticSearch. For the representation of the documents to the user as a search result on the page, some triples must be marked as representative of the result. Elasticsearch has the means of fuzzy comparison of terms that allows construction of relevant information search engines. The general technical aspects of the module functioning are discussed later in the application section.


\section{Implementation}
\label{sec:impl}

The transformation is implemented as a Logtalk library integrated with a RDF storage.  All the data processing is under control of SWI-Prolog.  In our research we suppose that the input of the process is the PIM is a set of logically linked UML-diagrams, ontologies and data stored in the warehouse of ontologies as well as other databases.  These data sources are assembled in a PIM with a data description scenario represented as process of PSM generation out of the sources data.

The output of the transformation is a set of configured interlinked Logtalk objects representing PIM.  The source codes of the subsystems and the initial states of databases are generated from these object complexes.

At this stage of the research we implemented a processor of input XMI-2 XML files, which contain the UML-model of an information system, instances of Logtalk classes.  The XMI files exported by Modelio UML designer consist of a model of the system as a set of top packages, and two profile packages.  One profile package describes local definitions, and another represents the system and code generation definitions.  The import processor converts the DOM tree of input XMI into corresponding graphs of triples.  Each triple represents one relationship between two structural elements of an UML diagram.  The instances have public interface used to query their graphs for the structural element compositions.

Platform model (PM) in a broad sense is the transformation procedure implemented as Logtalk components (instances) organized in an hierarchy of transformation modules in the image and likeness of our earlier work [].  Each component (module) queries the sources and constructs two classes of data: own state of private data and structures of external objects.  Own states in particular contain mappings of types between the source and target attributes, the implementation of designer decision on the general way of structure transformation.  The own states in fact represent the module configuration of its transformation state.

External objects are structures constructing PSM, i.e., the set or a category of configured objects reflecting PIM structures and target sources and data.  Transformation modules generate the objects as a results of general decisions.  Their configurations are also inferred from the properties of structural elements of PIM and the profile configurations.  The leaf nodes of the hierarchy is the scenario of PSM generation.  The usage of pure logical approach instead of the mix of two languages allowed us to reduce significantly the size and diversity of the PM description.

As soon as the scenario is fulfilled the PSM is converted into source codes and data.  The conversion in [] was made by means partial application of text templates.  In this case we do not use templates and generate the source code with simple procedures.  SWI-Prolong module index contains a simple template engine module \texttt{simple-template} intended for generation of static HTML pages.  The Prolog system also contain a dictionary-like structures and predicates for dictionary data manipulation, which can be used to represent data to be filled in the templates.  So open-source SWI-Prolog infrastructure is well equipped for developing MDA instruments.

\subsection{Inference of the class structure}
\label{sec:infstru}

Target class hierarchies in nowadays information system is diversified.  In web-applications, a PIM class or structure is mapped to a table or a SQL-query, an input form, an JavaScript object at user's browser, it can be a part of data transfer protocol like a JSON or XML object.  Some frameworks allow one to define structures and their relations and map them to rational databases [Annenkov], \emph{e.g.}, Django or Entity frameworks.  Most of the CASE-subsystems in UML composition tools do not support export of Class Diagram into the class structures.

The main difference of MDA with respect to CASE technologies is the possibility of adaptation of PD to the special way of software development used in a particular software development group.  To put it on another way, MDA implements the standard transformations and there is tools for modification of the standard procedures to implement own special techniques.  For example, if a programmer want to define a storage class for structures, the corresponding modification of standard transformation must be carried out.  Similarly a whole UML class hierarchy can be immersed into a Object-Relational Mapping (ORM) by means of a programmer-defined stereotype and its implementation.  Even an ORM itself may be implemented by means of special transformation, see [].


\subsection{Inference of the properties}
\label{sec:infprop}

One of the interesting application of the Semantic WEB technologies, namely, Linked Open Data ones is the inference of the properties of the generated structural elements.  The properties of PSM elements, \emph{e.g.}, instance attributes, are derived from PIM UML structure elements, \emph{i.e.}, their standard UML definitions, analysis of relations being involved in, features of combinations of associated stereotypes and tag values, as well as knowledge and data obtained from local ontology warehouse and SPARQL queries to other servers (that are cached for a better performance), for example, to DBPedia.org [].

For example, consider the user interface template generation for web or desktop application.  The model of the interface has a description of the structure and a specification of the properties of the constituent widgets.  If we refer from PIM with tag values \texttt{rdf:object} andd \texttt{rdf:type} of a special stereotype \texttt{<<dbpedia>>} to a corresponding DBPedia resource we can use the DBPedia subgraph for constructing \texttt{title} and \texttt{placeholder} attributes.  In principle, we can figure out a label of an input field for the user locale.

\subsection{Inference of methods}
\label{sec:infmeth}

Some UML-structures and relations may be interpreted constructively as methods of objects: events, handlers, subscribers and selectors.


\section{Applications}
\label{sec:app}

The described MDA tools are used and will be used in a number of interesting projects.  The first one was a Python driver of a cash register.

In the only Russian commercial planetary there appeared problem of connecting new cash register to a \emph{ad hoc} ticket and billing system.  The resource was limited to two weeks for general development and one programmer.  There was an old version of program for the previous cash machine, a documentation to the new one as well as test utility implementing the protocol obtained from vendor driver pack.  The description of the protocol in the manual was not precise: the sense of some fields were not understandable, the description of the automatons modeling the behavior in communications was too abstract, the documentation described two versions of the protocol, programmer has no previous experience with cash machines control.

The decision of application of the MDA was made after the first investigation of the situation.  The protocol commands and the automatons were represented as UML Class and State diagrams.

\section{Related research}
\label{sec:rel}

[[[[WSDL: Web Services Description Language]]]]

An XML based specification is used in the WS-BPEL language \cite{wsbpel} for describing orchestration of web services within execution of an instance of a business process.  The idea of the language is to describe complex systems down to Web Services units, their internal and external structure and behavior, as well as services interaction in the script instance. Our project concentrates on \emph{programming in the small} aspect of IS development, \emph{i.e.} dealing with modeling and implementation software components composing the services.


% \subsection{Subsection Heading Here}
% Subsection text here.


% \subsubsection{Subsubsection Heading Here}
% Subsubsection text here.


% An example of a floating figure using the graphicx package.
% Note that \label must occur AFTER (or within) \caption.
% For figures, \caption should occur after the \includegraphics.
% Note that IEEEtran v1.7 and later has special internal code that
% is designed to preserve the operation of \label within \caption
% even when the captionsoff option is in effect. However, because
% of issues like this, it may be the safest practice to put all your
% \label just after \caption rather than within \caption{}.
%
% Reminder: the "draftcls" or "draftclsnofoot", not "draft", class
% option should be used if it is desired that the figures are to be
% displayed while in draft mode.
%
%\begin{figure}[!t]
%\centering
%\includegraphics[width=2.5in]{myfigure}
% where an .eps filename suffix will be assumed under latex,
% and a .pdf suffix will be assumed for pdflatex; or what has been declared
% via \DeclareGraphicsExtensions.
%\caption{Simulation results for the network.}
%\label{fig_sim}
%\end{figure}

% Note that IEEE typically puts floats only at the top, even when this
% results in a large percentage of a column being occupied by floats.


% An example of a double column floating figure using two subfigures.
% (The subfig.sty package must be loaded for this to work.)
% The subfigure \label commands are set within each subfloat command,
% and the \label for the overall figure must come after \caption.
% \hfil is used as a separator to get equal spacing.
% Watch out that the combined width of all the subfigures on a
% line do not exceed the text width or a line break will occur.
%
%\begin{figure*}[!t]
%\centering
%\subfloat[Case I]{\includegraphics[width=2.5in]{box}%
%\label{fig_first_case}}
%\hfil
%\subfloat[Case II]{\includegraphics[width=2.5in]{box}%
%\label{fig_second_case}}
%\caption{Simulation results for the network.}
%\label{fig_sim}
%\end{figure*}
%
% Note that often IEEE papers with subfigures do not employ subfigure
% captions (using the optional argument to \subfloat[]), but instead will
% reference/describe all of them (a), (b), etc., within the main caption.
% Be aware that for subfig.sty to generate the (a), (b), etc., subfigure
% labels, the optional argument to \subfloat must be present. If a
% subcaption is not desired, just leave its contents blank,
% e.g., \subfloat[].


% An example of a floating table. Note that, for IEEE style tables, the
% \caption command should come BEFORE the table and, given that table
% captions serve much like titles, are usually capitalized except for words
% such as a, an, and, as, at, but, by, for, in, nor, of, on, or, the, to
% and up, which are usually not capitalized unless they are the first or
% last word of the caption. Table text will default to \footnotesize as
% IEEE normally uses this smaller font for tables.
% The \label must come after \caption as always.
%
%\begin{table}[!t]
%% increase table row spacing, adjust to taste
%\renewcommand{\arraystretch}{1.3}
% if using array.sty, it might be a good idea to tweak the value of
% \extrarowheight as needed to properly center the text within the cells
%\caption{An Example of a Table}
%\label{table_example}
%\centering
%% Some packages, such as MDW tools, offer better commands for making tables
%% than the plain LaTeX2e tabular which is used here.
%\begin{tabular}{|c||c|}
%\hline
%One & Two\\
%\hline
%Three & Four\\
%\hline
%\end{tabular}
%\end{table}


% Note that the IEEE does not put floats in the very first column
% - or typically anywhere on the first page for that matter. Also,
% in-text middle ("here") positioning is typically not used, but it
% is allowed and encouraged for Computer Society conferences (but
% not Computer Society journals). Most IEEE journals/conferences use
% top floats exclusively.
% Note that, LaTeX2e, unlike IEEE journals/conferences, places
% footnotes above bottom floats. This can be corrected via the
% \fnbelowfloat command of the stfloats package.




\section{Conclusion}





% conference papers do not normally have an appendix


% use section* for acknowledgment
\section*{Acknowledgment}

The results obtained with the partial support of the following projects:
\begin{itemize}
\item Irkutsk scientific center of SB RAS No 4.1.2;
\item The Council for grants of the President of Russian Federation, state support of leading scientific schools of the Russian Federation (NSH-8081.2016.9);
\item Russian Foundation for Basic Research (RFBR) No 17-07-01341.

  \end{itemize}
The results obtained with the use of the network infrastructure of Telecommunication center of collective use "Integrated information-computational network of Irkutsk scientific-educational complex" (\url{http://net.icc.ru}). The authors are grateful to the community of Linked Open Vocabularies (\url{http://lov.okfn.org/dataset/lov/}) resource for assistance in the search for formalizations of subject areas (ontologies).


% trigger a \newpage just before the given reference
% number - used to balance the columns on the last page
% adjust value as needed - may need to be readjusted if
% the document is modified later
%\IEEEtriggeratref{8}
% The "triggered" command can be changed if desired:
%\IEEEtriggercmd{\enlargethispage{-5in}}

% references section

% can use a bibliography generated by BibTeX as a .bbl file
% BibTeX documentation can be easily obtained at:
% http://www.ctan.org/tex-archive/biblio/bibtex/contrib/doc/
% The IEEEtran BibTeX style support page is at:
% http://www.michaelshell.org/tex/ieeetran/bibtex/
%\bibliographystyle{IEEEtran}
% argument is your BibTeX string definitions and bibliography database(s)
%\bibliography{IEEEabrv,../bib/paper}
%
% <OR> manually copy in the resultant .bbl file
% set second argument of \begin to the number of references
% (used to reserve space for the reference number labels box)
\begin{thebibliography}{11}

%\bibitem{IEEEhowto:kopka} H.~Kopka and P.~W. Daly, ``A Guide to \LaTeX,'' 3rd~ed.\hskip 1em plus 0.5em minus 0.4em\relax Harlow, England: Addison-Wesley, 1999.

\bibitem{Bizer} Ch. Bizer, T. Heath, T. Berners-Lee. ``Linked Data – The Story So Far,'' Semantic Web and Information Systems. 2009. Vol. 5 (3). pp. 1–22.
\bibitem{Cherk} E. Cherkashin, I. Orlova. ``Instrumental tools for construction of the digital archives of the documents based on Linked Data,'' Modern technologies, System analysis, Modeling. 4(56) 2017 pp. 100-107 (in Russian) \texttt{DOI:} \url{10.26731/1813-9108.2017.4(56).100-107}, \texttt{URL:} \url{http://stsam.irgups.ru/sites/default/files/articles\_pdf\_files/100-107.pdf}
\bibitem{Kopay} A. Kopaygorodsky. ``Use of ontologies in semantic information systems,'' Ontology of Design, 4(14), 2014, p. 78--89 (in Russian) \texttt{URL:} \href{http://agora.guru.ru/scientific_journal/files/Ontology_Of_Designing_4_2014_opt1.pdf#page=79}{\ttfamily http://agora.guru.ru/scientific\_journal/files/On\-tology\_Of\_Designing\_4\_2014\_opt1.pdf}
\bibitem{MDA} D. Frankel. Model Driven Architecture: Applying MDA to Enterprise Computing. Wiley; 1 edition, 2003, 352 p.
\bibitem{Clio} J. Wielemaker, W. Beek, M. Hildebrand, J. Ossenbruggen. ``ClioPatria: A SWI-Prolog Infrastructure for the Semantic Web,'' Semantic Web, vol. 7, no. 5, 2016, pp. 529-541. \texttt{DOI:} \url{10.3233/SW-150191}

%\bibitem{org} Org mode fantastic examples: \url{http://ehneilsen.net/notebook/orgExamples/org-examples.html}.
\bibitem{odmprof} ODM UML profile for OWL: \url{http://www.omg.org/spec/ODM/1.0/PDF/}.
\bibitem{odnex} OMG Ontology Domain Modeling example: \url{https://thematix.com/tools/vom/}.
\bibitem{odmvis} OWL UML Visualizer: \url{http://owlgred.lumii.lv/}.
\bibitem{uml2owd} UMLtoOWL: Converter from UML to OWL: \url{http://www.sfu.ca/~dgasevic/projects/UMLtoOWL/}. Uses XSLT to convert from XMI to OWL.
\bibitem{atmo3} AToM3: A tool for multi-paradigm modeling. \url{http://atom3.cs.mcgill.ca/}.
\bibitem{GT} Belghiat, Aissam \& Bourahla, Mustapha. (2012). UML Class Diagrams to OWL Ontologies: A Graph Transformation based Approach. International Journal of Computer Applications. 41. 41-46. 10.5120/5525-7566.
\bibitem{SWEB} \textbf{Semantic WEB Software Engineering}: \url{http://www.webist.org/Documents/Previous\_Invited\_Speakers/2012/WEBIST2012\_Pan.pdf}. Book: \url{https://www.iospress.nl/book/semantic-web-enabled-software-engineering/}.

\bibitem{Capadisli}  S.  Capadisli, S., Guy, A., Verborgh, R., Lange, C., Auer, S., Berners-Lee, T. ``Decentralised Authoring, Annotations and Notifications for a Read-Write Web with dokieli,'' In: Cabot J., De Virgilio R., Torlone R. (eds) Web Engineering. ICWE 2017. Lecture Notes in Computer Science, vol 10360. Springer, Cham. Preprint URL:\url{http://csarven.ca/dokieli-rww}, DOI:\url{10.1007/978-3-319-60131-1_33}.



\bibitem{cherk12}MDA is a complex system.
\bibitem{wsbpel} \url{https://en.wikipedia.org/wiki/Business_Process_Execution_Language}
\bibitem{wsdl} \url{https://en.wikipedia.org/wiki/Web_Services_Description_Language}
\bibitem{uml25} \url{http://www.omg.org/spec/UML/2.5/PDF}
\end{thebibliography}




% that's all folks
\end{document}

%%% Local Variables:
%%% TeX-master: t
%%% End:

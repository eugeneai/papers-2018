%% bare_conf.tex
%% V1.4a
%% 2014/09/17
%% by Michael Shell
%% See:
%% http://www.michaelshell.org/
%% for current contact information.
%%
%% This is a skeleton file demonstrating the use of IEEEtran.cls
%% (requires IEEEtran.cls version 1.8a or later) with an IEEE
%% conference paper.
%%
%% Support sites:
%% http://www.michaelshell.org/tex/ieeetran/
%% http://www.ctan.org/tex-archive/macros/latex/contrib/IEEEtran/
%% and
%% http://www.ieee.org/

%%*************************************************************************
%% Legal Notice:
%% This code is offered as-is without any warranty either expressed or
%% implied; without even the implied warranty of MERCHANTABILITY or
%% FITNESS FOR A PARTICULAR PURPOSE!
%% User assumes all risk.
%% In no event shall IEEE or any contributor to this code be liable for
%% any damages or losses, including, but not limited to, incidental,
%% consequential, or any other damages, resulting from the use or misuse
%% of any information contained here.
%%
%% All comments are the opinions of their respective authors and are not
%% necessarily endorsed by the IEEE.
%%
%% This work is distributed under the LaTeX Project Public License (LPPL)
%% ( http://www.latex-project.org/ ) version 1.3, and may be freely used,
%% distributed and modified. A copy of the LPPL, version 1.3, is included
%% in the base LaTeX documentation of all distributions of LaTeX released
%% 2003/12/01 or later.
%% Retain all contribution notices and credits.
%% ** Modified files should be clearly indicated as such, including  **
%% ** renaming them and changing author support contact information. **
%%
%% File list of work: IEEEtran.cls, IEEEtran_HOWTO.pdf, bare_adv.tex,
%%                    bare_conf.tex, bare_jrnl.tex, bare_conf_compsoc.tex,
%%                    bare_jrnl_compsoc.tex, bare_jrnl_transmag.tex
%%*************************************************************************


% *** Authors should verify (and, if needed, correct) their LaTeX system  ***
% *** with the testflow diagnostic prior to trusting their LaTeX platform ***
% *** with production work. IEEE's font choices and paper sizes can       ***
% *** trigger bugs that do not appear when using other class files.       ***                          ***
% The testflow support page is at:
% http://www.michaelshell.org/tex/testflow/



\documentclass[conference,a4paper]{IEEEtran}
% Some Computer Society conferences also require the compsoc mode option,
% but others use the standard conference format.
%
% If IEEEtran.cls has not been installed into the LaTeX system files,
% manually specify the path to it like:
% \documentclass[conference]{../sty/IEEEtran}

\usepackage[T2A]{fontenc}
\usepackage[utf8]{inputenc}
\usepackage[russian,mongolian,english]{babel}
\usepackage{tabularx,booktabs}
\newcolumntype{C}{>{\centering\arraybackslash}X} % centered version of "X" type
\setlength{\extrarowheight}{1pt}


% Some very useful LaTeX packages include:
% (uncomment the ones you want to load)


% *** MISC UTILITY PACKAGES ***
%
%\usepackage{ifpdf}
% Heiko Oberdiek's ifpdf.sty is very useful if you need conditional
% compilation based on whether the output is pdf or dvi.
% usage:
% \ifpdf
%   % pdf code
% \else
%   % dvi code
% \fi
% The latest version of ifpdf.sty can be obtained from:
% http://www.ctan.org/tex-archive/macros/latex/contrib/oberdiek/
% Also, note that IEEEtran.cls V1.7 and later provides a builtin
% \ifCLASSINFOpdf conditional that works the same way.
% When switching from latex to pdflatex and vice-versa, the compiler may
% have to be run twice to clear warning/error messages.






% *** CITATION PACKAGES ***
%
\usepackage{cite}
% cite.sty was written by Donald Arseneau
% V1.6 and later of IEEEtran pre-defines the format of the cite.sty package
% \cite{} output to follow that of IEEE. Loading the cite package will
% result in citation numbers being automatically sorted and properly
% "compressed/ranged". e.g., [1], [9], [2], [7], [5], [6] without using
% cite.sty will become [1], [2], [5]--[7], [9] using cite.sty. cite.sty's
% \cite will automatically add leading space, if needed. Use cite.sty's
% noadjust option (cite.sty V3.8 and later) if you want to turn this off
% such as if a citation ever needs to be enclosed in parenthesis.
% cite.sty is already installed on most LaTeX systems. Be sure and use
% version 5.0 (2009-03-20) and later if using hyperref.sty.
% The latest version can be obtained at:
% http://www.ctan.org/tex-archive/macros/latex/contrib/cite/
% The documentation is contained in the cite.sty file itself.

\usepackage{hyperref}
\hypersetup{
    % bookmarks=true,         % show bookmarks bar?
    unicode=true,          % non-Latin characters in Acrobat’s bookmarks
    pdftoolbar=true,        % show Acrobat’s toolbar?
    pdfmenubar=true,        % show Acrobat’s menu?
    pdffitwindow=false,     % window fit to page when opened
    pdfstartview={FitH},    % fits the width of the page to the window
    %pdftitle={},    % title
    %pdfauthor={Author},     % author
    %pdfsubject={Subject},   % subject of the document
    %pdfcreator={Creator},   % creator of the document
    %pdfproducer={Producer}, % producer of the document
    %pdfkeywords={keyword1, key2, key3}, % list of keywords
    %pdfnewwindow=true,      % links in new PDF window
    colorlinks=true,       % false: boxed links; true: colored links
    linkcolor=black,          % color of internal links (change box color with linkbordercolor)
    citecolor=black,        % color of links to bibliography
    filecolor=black,      % color of file links
    urlcolor=black,           % color of external links
    final=true
  }




% *** GRAPHICS RELATED PACKAGES ***
%
\ifCLASSINFOpdf
  % \usepackage[pdftex]{graphicx}
  % declare the path(s) where your graphic files are
  % \graphicspath{{../pdf/}{../jpeg/}}
  % and their extensions so you won't have to specify these with
  % every instance of \includegraphics
  % \DeclareGraphicsExtensions{.pdf,.jpeg,.png}
\else
  % or other class option (dvipsone, dvipdf, if not using dvips). graphicx
  % will default to the driver specified in the system graphics.cfg if no
  % driver is specified.
  % \usepackage[dvips]{graphicx}
  % declare the path(s) where your graphic files are
  % \graphicspath{{../eps/}}
  % and their extensions so you won't have to specify these with
  % every instance of \includegraphics
  % \DeclareGraphicsExtensions{.eps}
\fi
% graphicx was written by David Carlisle and Sebastian Rahtz. It is
% required if you want graphics, photos, etc. graphicx.sty is already
% installed on most LaTeX systems. The latest version and documentation
% can be obtained at:
% http://www.ctan.org/tex-archive/macros/latex/required/graphics/
% Another good source of documentation is "Using Imported Graphics in
% LaTeX2e" by Keith Reckdahl which can be found at:
% http://www.ctan.org/tex-archive/info/epslatex/
%
% latex, and pdflatex in dvi mode, support graphics in encapsulated
% postscript (.eps) format. pdflatex in pdf mode supports graphics
% in .pdf, .jpeg, .png and .mps (metapost) formats. Users should ensure
% that all non-photo figures use a vector format (.eps, .pdf, .mps) and
% not a bitmapped formats (.jpeg, .png). IEEE frowns on bitmapped formats
% which can result in "jaggedy"/blurry rendering of lines and letters as
% well as large increases in file sizes.
%
% You can find documentation about the pdfTeX application at:
% http://www.tug.org/applications/pdftex





% *** MATH PACKAGES ***
%
%\usepackage[cmex10]{amsmath}
% A popular package from the American Mathematical Society that provides
% many useful and powerful commands for dealing with mathematics. If using
% it, be sure to load this package with the cmex10 option to ensure that
% only type 1 fonts will utilized at all point sizes. Without this option,
% it is possible that some math symbols, particularly those within
% footnotes, will be rendered in bitmap form which will result in a
% document that can not be IEEE Xplore compliant!
%
% Also, note that the amsmath package sets \interdisplaylinepenalty to 10000
% thus preventing page breaks from occurring within multiline equations. Use:
%\interdisplaylinepenalty=2500
% after loading amsmath to restore such page breaks as IEEEtran.cls normally
% does. amsmath.sty is already installed on most LaTeX systems. The latest
% version and documentation can be obtained at:
% http://www.ctan.org/tex-archive/macros/latex/required/amslatex/math/





% *** SPECIALIZED LIST PACKAGES ***
%
%\usepackage{algorithmic}
% algorithmic.sty was written by Peter Williams and Rogerio Brito.
% This package provides an algorithmic environment fo describing algorithms.
% You can use the algorithmic environment in-text or within a figure
% environment to provide for a floating algorithm. Do NOT use the algorithm
% floating environment provided by algorithm.sty (by the same authors) or
% algorithm2e.sty (by Christophe Fiorio) as IEEE does not use dedicated
% algorithm float types and packages that provide these will not provide
% correct IEEE style captions. The latest version and documentation of
% algorithmic.sty can be obtained at:
% http://www.ctan.org/tex-archive/macros/latex/contrib/algorithms/
% There is also a support site at:
% http://algorithms.berlios.de/index.html
% Also of interest may be the (relatively newer and more customizable)
% algorithmicx.sty package by Szasz Janos:
% http://www.ctan.org/tex-archive/macros/latex/contrib/algorithmicx/




% *** ALIGNMENT PACKAGES ***
%
%\usepackage{array}
% Frank Mittelbach's and David Carlisle's array.sty patches and improves
% the standard LaTeX2e array and tabular environments to provide better
% appearance and additional user controls. As the default LaTeX2e table
% generation code is lacking to the point of almost being broken with
% respect to the quality of the end results, all users are strongly
% advised to use an enhanced (at the very least that provided by array.sty)
% set of table tools. array.sty is already installed on most systems. The
% latest version and documentation can be obtained at:
% http://www.ctan.org/tex-archive/macros/latex/required/tools/


% IEEEtran contains the IEEEeqnarray family of commands that can be used to
% generate multiline equations as well as matrices, tables, etc., of high
% quality.




% *** SUBFIGURE PACKAGES ***
%\ifCLASSOPTIONcompsoc
%  \usepackage[caption=false,font=normalsize,labelfont=sf,textfont=sf]{subfig}
%\else
%  \usepackage[caption=false,font=footnotesize]{subfig}
%\fi
% subfig.sty, written by Steven Douglas Cochran, is the modern replacement
% for subfigure.sty, the latter of which is no longer maintained and is
% incompatible with some LaTeX packages including fixltx2e. However,
% subfig.sty requires and automatically loads Axel Sommerfeldt's caption.sty
% which will override IEEEtran.cls' handling of captions and this will result
% in non-IEEE style figure/table captions. To prevent this problem, be sure
% and invoke subfig.sty's "caption=false" package option (available since
% subfig.sty version 1.3, 2005/06/28) as this is will preserve IEEEtran.cls
% handling of captions.
% Note that the Computer Society format requires a larger sans serif font
% than the serif footnote size font used in traditional IEEE formatting
% and thus the need to invoke different subfig.sty package options depending
% on whether compsoc mode has been enabled.
%
% The latest version and documentation of subfig.sty can be obtained at:
% http://www.ctan.org/tex-archive/macros/latex/contrib/subfig/




% *** FLOAT PACKAGES ***
%
%\usepackage{fixltx2e}
% fixltx2e, the successor to the earlier fix2col.sty, was written by
% Frank Mittelbach and David Carlisle. This package corrects a few problems
% in the LaTeX2e kernel, the most notable of which is that in current
% LaTeX2e releases, the ordering of single and double column floats is not
% guaranteed to be preserved. Thus, an unpatched LaTeX2e can allow a
% single column figure to be placed prior to an earlier double column
% figure. The latest version and documentation can be found at:
% http://www.ctan.org/tex-archive/macros/latex/base/


%\usepackage{stfloats}
% stfloats.sty was written by Sigitas Tolusis. This package gives LaTeX2e
% the ability to do double column floats at the bottom of the page as well
% as the top. (e.g., "\begin{figure*}[!b]" is not normally possible in
% LaTeX2e). It also provides a command:
%\fnbelowfloat
% to enable the placement of footnotes below bottom floats (the standard
% LaTeX2e kernel puts them above bottom floats). This is an invasive package
% which rewrites many portions of the LaTeX2e float routines. It may not work
% with other packages that modify the LaTeX2e float routines. The latest
% version and documentation can be obtained at:
% http://www.ctan.org/tex-archive/macros/latex/contrib/sttools/
% Do not use the stfloats baselinefloat ability as IEEE does not allow
% \baselineskip to stretch. Authors submitting work to the IEEE should note
% that IEEE rarely uses double column equations and that authors should try
% to avoid such use. Do not be tempted to use the cuted.sty or midfloat.sty
% packages (also by Sigitas Tolusis) as IEEE does not format its papers in
% such ways.
% Do not attempt to use stfloats with fixltx2e as they are incompatible.
% Instead, use Morten Hogholm'a dblfloatfix which combines the features
% of both fixltx2e and stfloats:
%
% \usepackage{dblfloatfix}
% The latest version can be found at:
% http://www.ctan.org/tex-archive/macros/latex/contrib/dblfloatfix/




% *** PDF, URL AND HYPERLINK PACKAGES ***
%
%\usepackage{url}
% url.sty was written by Donald Arseneau. It provides better support for
% handling and breaking URLs. url.sty is already installed on most LaTeX
% systems. The latest version and documentation can be obtained at:
% http://www.ctan.org/tex-archive/macros/latex/contrib/url/
% Basically, \url{my_url_here}.



% *** Do not adjust lengths that control margins, column widths, etc. ***
% *** Do not use packages that alter fonts (such as pslatex).         ***
% There should be no need to do such things with IEEEtran.cls V1.6 and later.
% (Unless specifically asked to do so by the journal or conference you plan
% to submit to, of course. )


% correct bad hyphenation here
\hyphenation{op-tical net-works semi-conduc-tor}

\providecommand\url[1]{\texttt{#1}}

\newcommand{\specialcell}[2][c]{%
  \begin{tabular}[#1]{@{}c@{}}#2\end{tabular}}

\begin{document}
\urlstyle{tt}

%
% paper title
% Titles are generally capitalized except for words such as a, an, and, as,
% at, but, by, for, in, nor, of, on, or, the, to and up, which are usually
% not capitalized unless they are the first or last word of the title.
% Linebreaks \\ can be used within to get better formatting as desired.
% Do not put math or special symbols in the title.
\date{}
\title{Phonetic String Matching in Russian Language}

%\author{Evgeny Cherkashin, Alexey Kopaigorodsky, Ljubica Kazi.}

% author names and affiliations
% use a multiple column layout for up to three different
% affiliations
% \author{\IEEEauthorblockN{Michael Shell}
% \IEEEauthorblockA{School of Electrical and\\Computer Engineering\\
% Georgia Institute of Technology\\
% Atlanta, Georgia 30332--0250\\
% Email: http://www.michaelshell.org/contact.html}
% \and
% \IEEEauthorblockN{Homer Simpson}
% \IEEEauthorblockA{Twentieth Century Fox\\
% Springfield, USA\\
% Email: homer@thesimpsons.com}
% \and
% \IEEEauthorblockN{James Kirk\\ and Montgomery Scott}
% \IEEEauthorblockA{Starfleet Academy\\
% San Francisco, California 96678--2391\\
% Telephone: (800) 555--1212\\
% Fax: (888) 555--1212}}

% conference papers do not typically use \thanks and this command
% is locked out in conference mode. If really needed, such as for
% the acknowledgment of grants, issue a \IEEEoverridecommandlockouts
% after \documentclass

% for over three affiliations, or if they all won't fit within the width
% of the page, use this alternative format:
%

\DeclareRobustCommand*{\IEEEauthorrefmark}[1]{\raisebox{0pt}[0pt][0pt]{\textsuperscript{\footnotesize #1}}}

\author{\IEEEauthorblockN{Viacheslav Paramonov\IEEEauthorrefmark{1,}\IEEEauthorrefmark{2},
Alexey Shigarov\IEEEauthorrefmark{1,}\IEEEauthorrefmark{2},
Gennady Ruzhnikov\IEEEauthorrefmark{1},
Evgeny Cherkashin\IEEEauthorrefmark{1,}\IEEEauthorrefmark{2,}\IEEEauthorrefmark{3}}\\[-0.9em]
%
\IEEEauthorblockA{\IEEEauthorrefmark{1}Matrosov Institute for System Dynamics and Control Theory of SB RAS, 134 Lermontov Street, Russia, 664033}
\IEEEauthorblockA{\IEEEauthorrefmark{2}Institute of Mathematics, Economics and Informatics \\ Irkutsk State University, 20 Gagarina Avenue, Russia, 664003}
\IEEEauthorblockA{\IEEEauthorrefmark{3}National Research Irkutsk State Technical University, 83 Lermontov Street, Russia, 664074}
E-mail:\texttt{\{slv, shigarov, rugnikov, eugeneai\}@icc.ru}}







% use for special paper notices
%\IEEEspecialpapernotice{(Invited Paper)}




% make the title area
\maketitle

% As a general rule, do not put math, special symbols or citations
% in the abstract
\begin{abstract}
The usage of phonetic similarity for word comparison and elimination of misprints is a debatable issue in philology. Moreover, this is very actual for using in automatic text processing. Most of the currently used phonetic algorithms are designed for for automatic processing of English language. The accuracy of English-specific techniques descends for other natural languages with rich morphology and non-Latin alphabet symbols, e.g. Slavic languages with Cyrillic alphabets. The work is devoted to design of phonetic algorithm that does not use transliteration and considers particular qualities of Russian language. The method is based on detection letters and letter consequences that have similar sound. These symbols transform to some generic values which are coding by primes. The results of experimental comparison of proposed algorithm with other well-known phonetic algorithms show that the proposed algorithm can be used to improve an accuracy of word comparison for Cyrillic languages in tasks of data cleansing. An adaptation of the proposed algorithm for Mongolian language is also presented.
\end{abstract}

%%%% KEYWORDS spreadsheet data transformation, table analysis, table interpretation, rule-based systems

% no keywords




% For peer review papers, you can put extra information on the cover
% page as needed:
% \ifCLASSOPTIONpeerreview
% \begin{center} \bfseries EDICS Category: 3-BBND \end{center}
% \fi
%
% For peerreview papers, this IEEEtran command inserts a page break and
% creates the second title. It will be ignored for other modes.
\IEEEpeerreviewmaketitle



\section{Introduction}
% no \IEEEPARstart
Digital data integration is one of the popular areas of information society. It is the part of business intelligence. Integration of different documents which have different formats of data representation allow to accumulate large dataset of information and use it for different criteria of analysis. In common this is accord with ETL-processes.

Initial data could be collect and forming from different sources. As a consequence of this integrable data are heterogeneous, represented in different formats and may have some errors. They arise as a result of misprints, spelling mistakes, invalid usage of date or currency delimiters, errors of text recognition etc. In this regard, data cleanse is required before the integration.

Data cleansing is one of important steps in data integration processes. Data cleansing includes many aspects such as detection and automatic correction of spelling errors, incorrect values coding and logical inconsistencies, missing data. This paper consider one of messy data kinds -- detection and correction words with applying methods of phonetic comparison.

Some of integrable data are textual data which could be linked with dictionaries (classifiers) values. For example in data about types and location places of vegetation and spiders in region. Some dataset exists in Scientific Institution's data bases, some data collected by scientists individually.

In different countries data may be represented on different, national languages. In this paper textual data on Russian language are reviewed. Russian language belongs to East-Slavic language group. It is used by more than 250 million speakers \cite{Cubberley-2002}.

There are a lot of methods to word comparison with values of it represented in dictionaries. One of the methods of errors detection is based on words phonetic codes comparison. The paper introduces the method of detection and correction of Russian language spelling errors in data normalization processes. The algorithm is based on phonetic coded strings matching analysis.

The idea of phonetic algorithms is seeking identify words according to their similar pronunciation \cite{Parmar-2014}. The common purpose of phonetic algorithms is detection of the words similarity according to their phonetic resemblance. The most typical application of phonetic algorithms is intended for the surnames \cite{Zahoransky-2015}. However it is possible to use phonetic algorithms for verification words that are other parts of speech. Possibility of it depend on language features and particularities. This paper devoted the usage of phonetic algorithm for data cleanse tasks in Russian language and adoption it for Mongolian language.

\section{Data anomalies}
Textual data that represented in official databases have very high level of correctness. These data are passed a lot of verification procedures. Another situation with data which were collected form persons for their own usage. These data may contains some errors – anomalies. Often they may arise in processes of data entry. The rate of these errors is depend on many criteria and it is about 5\% for educated adult person \cite{Orr-1998}.

In the \cite{Osipov-2002} allocate such anomalies as: coverage, semantic and syntactical. To improve the quality of data all kinds of errors is should be minimized. Coverage and semantic anomalies could not be eliminate with applying methods of phonetic comparison. However these methods may fix some of syntactical errors.

These type of anomalies include lexical errors, which relates with name discrepancies between the structure of the data items and the specified format –- integrable data do not correspond to the data from dictionaries.  Domain format of anomalies that associated with correct data do not conform to the anticipated domain format – the absence of separator in integrable data. The letters admission or their incorrect data also it is possible to carry to this type of mistakes.

Irregularities are concerned with the non-uniform use of values, units and abbreviations.
We suggest that phonetic algorithms may be helpful in eliminating of syntactical such as lexical errors. Some of these kind anomalies are result of various typos and words doubles in process of word typing. Some of typos occur due phonetic errors.

\section{Usage phonetic algorithms for Russian text cleansing tasks}
\subsection{Phonetic spelling errors}
Types of spelling errors depend on language particularities. For example spelling errors of Russian language words may be subject to following principles \cite{Skripnik-2010}:
\begin{itemize}
\item \textbf{morphological} – a uniform graphic symbol of morphemes by the letter i.e. a person tries to write all audible sounds by letters \cite{Valgina-2002};
\item \textbf{phonologic} – preservation of writing of phonemes spelling regardless the word of change;
\item \textbf{phonetic} – words written as they are heard;
\item \textbf{traditional} (historic) – writing by “tradition” i.e. as it was written in old times or as in the language which it is borrowed from
\end{itemize}

Most of spelling errors in Russian are associated with phonetic norms. The type of mistakes generally depends on the level of one’s education \cite{Parubchenko-2005}. For example, elementary school students make mistakes because they write words as heard. High school students are more prone to hypercorrection errors. As the literacy level of a person does not improve after school graduation, adult persons also make hypercorrection errors. However, in general, many spelling mistakes are simply typos or associated with phonetic words representation.

\begin{table}[t!]
\renewcommand{\arraystretch}{1.3}
\caption{Share of most various spelling patterns in total number of mistakes.}
\label{lbl-mistakes}
\centering
\begin{tabular}{|c|c|c|}
\hline Place & Spelling pattern type & \specialcell{Share, \\ \%}\\
\hline 1 & 	Mixing of "и"-"е" in unstressed  syllable	& 27.3 \\
\hline 2 &	Mixing "а"-"о" in  unstressed syllable	& 25.3 \\
\hline 3 &	Separate writing instead of solid word writing	& 9.4 \\
\hline 4 &	Solid word writing instead of the separate &	8.8 \\
\hline 5 &	Writing of one letter instead of the doubled &	6.6 \\
\hline 6 &	Mixture deaf and ringing letters &	3.6 \\
\hline 7 &	Vowels after hissing sounds and ц &	2.7 \\
\hline 8 &	Excess doubling of letters &	2.6 \\
\hline 9 &	Absence of "ь" &	1.3 \\
\hline 10 &	Writing of the superfluous "ь" &	0.6 \\
\hline
\end{tabular}
\end{table}

It is noted in \cite{Osipov-2002} that phonetic spelling principle is the most natural and efficient. Therefore  phonetic algorithms would be useful for similar strings matching. Table \ref{lbl-mistakes} represents Russian language general spelling errors. It is a simplified table from \cite{Osipov-2002}. Here and in other samples Russian language words would be presented in transliteration also according \cite{GOST-2006} and shown in \textbf{[ ]} characters. This designation will help to facilitate the understanding of text for persons who are not aware of Cyrillic characters.

Thus, these kinds of spelling errors make nearly 90\% of all errors. Other spelling errors are represented by mixing of some vowels, concordant, letter sequences or dividing (special) letters “\textbf{ь}” ("soft sign") and “\textbf{ъ}” ("hard sign"). Forming rules that allow to find words matching with due regard to errors type help to find and eliminate errors in words.

\subsection{Phonetic algorithms}
The general idea of phonetic algorithms is words comparison according their pronunciation (phonetic forms) and does not depend on orthographical rules. Words are phonetically similar if their codes are matching.

This kind of algorithms allow to find typos related to changing places of two adjacent letters and typos based on sound similarities. We suggest that the following categories of errors are allocated to estimate overall performance of phonetic algorithms:
\begin{itemize}
\item \textbf{first category} – errors related with incorrect writing (morphological errors);
\item \textbf{second category} – various typographical errors (misprints, typos);
\item \textbf{third category} – errors related to incorrect rule usage (hypercorrection \cite{Parubchenko-2005} errors).
\end{itemize}

Well-known algorithms based on English-words coding. Some algorithms modification for Turkish, Spanish language described in \cite{Alotaibi-2013}. However this languages are based on Latin alphabet symbols. It is exists some of phonetic algorithms for other languages where alphabet characters differ than English. As rule this approach based symbols transliteration and on Soundex phonetic algorithm \cite{Soundex}.

Applying of well-known phonetic algorithms for Russian text is a result of transliterated Cyrillic to Latin characters coding. However transliterated words do not always have a unique record. For example, such name as “Антон Чехов” [“Anton Chekhov”] has been variously transliterated as Tsjechof, Tsjechow, Tjekhow, Chekhov, Chekhow etc. 

Usage of phonetic codes allows increasing the word comparison quality in the case of their incorrect writing. All phonetic algorithms use words coding. Changing of noun case and form, for example, leads to ineffective use of phonetic algorithms. However, such changes are not significant in our case. Therefore phonetic algorithms are most suitable for word comparison with reference books or dictionaries. Phonetic algorithms are used expediently for the solution of issues of comparison word with their meanings in reference books or dictionaries (classifiers).

\section{Russian language phonetic string matching approach}
Transliteration allows to get text in Cyrillic letters in Latin letters representation. It should be noted that there are no standard rules of transliteration that take into account peculiar properties of a language and it does not depend on language in East--Slavic language group. For example, Ukrainian word "вчора" (yesterday) we can transliterate as "vchera", "vchora", "fchora". In addition, misspellings in East – Slavic languages with Cyrillic letters generally differ from these in English or German texts. The reasons for it are different rules of pronunciation and writing in different languages. In transliteration it is unable to consider features of letter sequences for each language.

The suggested in the paper algorithm Polyphon makes used word transformation with due regard to rules of Russian language and according to its phonetic particularities. It is allows get a more correct phonetic code for conformable strings. The stages of algorithm are:

\begin{enumerate}
\item substitution of Latin letters which are similar to Russian with Russian letters;
\item removal of all non-Russian alphabet characters from the string;
\item modification of letters before dividers (special letters as "ъ" and "ь");
\item removal dividers from string;
\item transformation of similar letters following each other into one;
\item transformation of character sequences.
\end{enumerate}

Let us consider algorithm steps more detail. Some of letter in Russian alphabet have equivalent in writing with Latin. These are such letters as: a \textbf{[a]}$\sim$ а, e \textbf{[e]} $\sim$ е, о \textbf{[o]} $\sim$ о, c \textbf{[es]} $\sim$ с, x \textbf{[kha]} $\sim$ x. Some letters equal in capital letters only: B \textbf{[ve]} $\sim$ В, M \textbf{[em]} $\sim$ М, H \textbf{[en]} $\sim$ Н. Sometimes these letters are substituted (incidentally or purposely) when text typing. 

There are some dividers – special letters “ь” (“soft” sign), “ъ” (“hard” sing) presented in Russian language. They are not pronounce and are used for giving softness or hardness for consonants respectively. For this reason there is no need to consider these characters.

The offered phonetic algorithm Polyphon operates with Russian language letters only. The initial operation is to remove all characters, which do not use in Russian alphabet. 

The following stage is transformation of letters repeated in a row. Doubled letters will transform to one e.g. “xx” to “x”. It is not always possible to define double letters in hearing. Therefore, we carry out these transformations for rule of generalization.

It is taken into account that phonetic code depends on the sound that can come from some letters or their sequences. Some different letters or their combinations have different sounds. Thus, Polyphone uses coding letters by sounds, which are heard. Ways of the replacement are provided in table III. The aim of the proposed phonetic algorithm generalized letters and sounds combinations [18]. The reason of generalization is based on that some sounds from letter sequences depend on stress position.  Such deviations from norms are meet in social and territorial dialects in Russia \cite{Zhirmunsky-1936}.


\subsection{Subsection Heading Here}
Subsection text here.


\subsubsection{Subsubsection Heading Here}
Subsubsection text here.


\begin{table*}
 \caption{CIFAR-10 Confusion Matrix}
\label{my-label}
\begin{tabularx}{\textwidth}{@{}l*{10}{C}c@{}}
\toprule
labels     & airplane & automobile & bird & cat & deer & dog & frog & horse & ship & truck & accuracy \\ 
\midrule
airplane   & 915      & 4          & 17   & 19  & 3    & 1   & 0    & 2     & 27   & 12    & 91.50\%  \\ 
automobile & 8        & 934        & 3    & 4   & 0    & 0   & 3    & 0     & 10   & 38    & 93.40\%  \\ 
bird       & 60       & 1          & 813  & 37  & 19   & 23  & 30   & 10    & 7    & 0     & 81.30\%  \\ 
cat        & 18       & 1          & 34   & 746 & 25   & 113 & 37   & 18    & 8    & 0     & 74.60\%  \\ 
deer       & 24       & 1          & 38   & 33  & 809  & 19  & 44   & 29    & 2    & 1     & 80.90\%  \\ 
\addlinespace
dog        & 4        & 0          & 37   & 106 & 23   & 792 & 9    & 26    & 2    & 1     & 79.20\%  \\ 
frog       & 2        & 5          & 19   & 35  & 1    & 20  & 912  & 2     & 3    & 1     & 91.20\%  \\ 
horse      & 14       & 0          & 26   & 20  & 18   & 28  & 4    & 886   & 3    & 1     & 88.60\%  \\ 
ship       & 35       & 10         & 3    & 2   & 0    & 2   & 1    & 0     & 936  & 11    & 93.60\%  \\ 
truck      & 23       & 37         & 4    & 10  & 1    & 2   & 2    & 0     & 15   & 906   & 90.60\%  \\ 
\bottomrule
\end{tabularx}
\end{table*}

% An example of a floating figure using the graphicx package.
% Note that \label must occur AFTER (or within) \caption.
% For figures, \caption should occur after the \includegraphics.
% Note that IEEEtran v1.7 and later has special internal code that
% is designed to preserve the operation of \label within \caption
% even when the captionsoff option is in effect. However, because
% of issues like this, it may be the safest practice to put all your
% \label just after \caption rather than within \caption{}.
%
% Reminder: the "draftcls" or "draftclsnofoot", not "draft", class
% option should be used if it is desired that the figures are to be
% displayed while in draft mode.
%
%\begin{figure}[!t]
%\centering
%\includegraphics[width=2.5in]{myfigure}
% where an .eps filename suffix will be assumed under latex,
% and a .pdf suffix will be assumed for pdflatex; or what has been declared
% via \DeclareGraphicsExtensions.
%\caption{Simulation results for the network.}
%\label{fig_sim}
%\end{figure}

% Note that IEEE typically puts floats only at the top, even when this
% results in a large percentage of a column being occupied by floats.


% An example of a double column floating figure using two subfigures.
% (The subfig.sty package must be loaded for this to work.)
% The subfigure \label commands are set within each subfloat command,
% and the \label for the overall figure must come after \caption.
% \hfil is used as a separator to get equal spacing.
% Watch out that the combined width of all the subfigures on a
% line do not exceed the text width or a line break will occur.
%
%\begin{figure*}[!t]
%\centering
%\subfloat[Case I]{\includegraphics[width=2.5in]{box}%
%\label{fig_first_case}}
%\hfil
%\subfloat[Case II]{\includegraphics[width=2.5in]{box}%
%\label{fig_second_case}}
%\caption{Simulation results for the network.}
%\label{fig_sim}
%\end{figure*}
%
% Note that often IEEE papers with subfigures do not employ subfigure
% captions (using the optional argument to \subfloat[]), but instead will
% reference/describe all of them (a), (b), etc., within the main caption.
% Be aware that for subfig.sty to generate the (a), (b), etc., subfigure
% labels, the optional argument to \subfloat must be present. If a
% subcaption is not desired, just leave its contents blank,
% e.g., \subfloat[].


% An example of a floating table. Note that, for IEEE style tables, the
% \caption command should come BEFORE the table and, given that table
% captions serve much like titles, are usually capitalized except for words
% such as a, an, and, as, at, but, by, for, in, nor, of, on, or, the, to
% and up, which are usually not capitalized unless they are the first or
% last word of the caption. Table text will default to \footnotesize as
% IEEE normally uses this smaller font for tables.
% The \label must come after \caption as always.
%
%\begin{table}[!t]
%% increase table row spacing, adjust to taste
%\renewcommand{\arraystretch}{1.3}
% if using array.sty, it might be a good idea to tweak the value of
% \extrarowheight as needed to properly center the text within the cells
%\caption{An Example of a Table}
%\label{table_example}
%\centering
%% Some packages, such as MDW tools, offer better commands for making tables
%% than the plain LaTeX2e tabular which is used here.
%\begin{tabular}{|c||c|}
%\hline
%One & Two\\
%\hline
%Three & Four\\
%\hline
%\end{tabular}
%\end{table}


% Note that the IEEE does not put floats in the very first column
% - or typically anywhere on the first page for that matter. Also,
% in-text middle ("here") positioning is typically not used, but it
% is allowed and encouraged for Computer Society conferences (but
% not Computer Society journals). Most IEEE journals/conferences use
% top floats exclusively.
% Note that, LaTeX2e, unlike IEEE journals/conferences, places
% footnotes above bottom floats. This can be corrected via the
% \fnbelowfloat command of the stfloats package.




\section{Conclusion}
The conclusion goes here.




% conference papers do not normally have an appendix


% use section* for acknowledgment
\section*{Acknowledgement}
The reported study was supported in part by RFBR (grants 18-07-00758, 17-47-380007). Experiments were performed on the resources of the Shared Equipment Centre of Integrated information and computing network of Irkutsk Research and Educational Complex (http://net.icc.ru).





% trigger a \newpage just before the given reference
% number - used to balance the columns on the last page
% adjust value as needed - may need to be readjusted if
% the document is modified later
%\IEEEtriggeratref{8}
% The "triggered" command can be changed if desired:
%\IEEEtriggercmd{\enlargethispage{-5in}}

% references section

% can use a bibliography generated by BibTeX as a .bbl file
% BibTeX documentation can be easily obtained at:
% http://www.ctan.org/tex-archive/biblio/bibtex/contrib/doc/
% The IEEEtran BibTeX style support page is at:
% http://www.michaelshell.org/tex/ieeetran/bibtex/
%\bibliographystyle{IEEEtran}
% argument is your BibTeX string definitions and bibliography database(s)
%\bibliography{IEEEabrv,../bib/paper}
%
% <OR> manually copy in the resultant .bbl file
% set second argument of \begin to the number of references
% (used to reserve space for the reference number labels box)
\begin{thebibliography}{11}

\bibitem{Cubberley-2002} P.~Cubberley Russian: A Linguistic Introduction. Cambridge press. --– 2002. 396 p.
\bibitem{Parmar-2014}	V.P.~Parmar, C.K.~Kumbharana. Study Existing Various Phonetic Algorithms and Designing and Development of a working model for the New Developed Algorithm and Comparison by implementing it with Existing Algorithm(s) // International Journal of Computer Applications (0975 –-- 8887) Volume 98 -– No.19, July 2014. --– pp. 45--49.
\bibitem{Zahoransky-2015} D.~Zahoransky, I.~Polasek. Text Search of Surnames in Some Slavic and Other Morphologically Rich Languages Using Rule Based Phonetic Algorithms // Audio, Speech, and Language Processing, IEEE/ACM Trans on (T--ASL). IEEE. --– 2015. pp.553--563
\bibitem{Orr-1998} K.~Orr. Data Quality and Systems Theory. // Communications of the ACM, Vol. 41, No. 2, 1998, 66--71
\bibitem{Osipov-2002}	B.I.~Osipov, L.G.~Galushinskaya, V.V.~Popkov. Phonetic and hyper-correction errors in written assignments of students of 3--11 classes of high school (Фонетические и гиперические ошибки в письменных работах учащихся 3–11-х классов средней школы). Russian Language journal. No 15, 2002 (in Russian) \texttt{URL:} \url{http://rus.1september.ru/article.php?ID=200201501}
\bibitem{Parubchenko-2005} L.B.~Parubchenko. Hypercorrection errors (Ошибки гиперкоррекции)// Russian Literature. No 4, 2005.-- p.p. 23--27. (in Russian)
\bibitem{Skripnik-2010} Ya.N.~Skripnik, T.M.~Smolenskaya. Phonetics of modern Russian Language (Фонетика современного русского языка: Учебное пособие) / Editor: Ya.N.~Skripnik. -–- Stavropol --- VoSIGI. --- 2010. --– 152~p. (in Russian)
\bibitem{Valgina-2002} N.S.~Valgina, D.E.~Rozental', M.I.Fomina. Modern Russian Language: Textbook (Современный русский язык: Учебник) / Editor: N.S.~Valgina -- 6-th Edition. --- Moscow --- Logos --- 2002 --– 528~p. (in Russian)
\bibitem{GOST-2006} GOST R 52535.1-2006. Identification cards. Machine readable travel documents. Part 1 Machine Readable Passports. National Standard of the Russian Federation (ГОСТ Р 52535.1-2006. Карты идентификационные. Машиносчитываемые дорожные документы. Часть 1. Машиносчитываемые паспорта. Национальный стандарт Российской Федерации). --- Moscow --- Russia. --- 2006 --- 18 p. (in Russian)
\bibitem{Alotaibi-2013} Y.~Alotaibi, A.~Meftah. Review of distinctive phonetic features and the Arabic share in related modern research // Turkish Journal of Electrical Engineering \& Computer Sciences 2013, Vol. 21 Issue 5, pp.~1426-1439
\bibitem{Soundex}	The Soundex Indexing System. National archives. \texttt{URL:} \url{http://www.archives.gov/research/census/soundex.html}
\bibitem{Zhirmunsky-1936} V.~Zhirmunsky. National Language and social dialects (Национальный язык и социальные диалекты). --- Moscow: The state publisher of fiction. --- 1936. --- 300 p. (in Russian)





\bibitem{Bizer} Ch. Bizer, T. Heath, T. Berners-Lee. ``Linked Data – The Story So Far,'' Semantic Web and Information Systems. 2009. Vol. 5 (3). pp. 1–22.
\bibitem{Cherk} E. Cherkashin, I. Orlova. Instrumental tools for construction of the digital archives of the documents based on Linked Data. Modern technologies, System analysis, Modeling. 4(56) 2017 pp. 100-107 (in Russian) \texttt{DOI:} \url{10.26731/1813-9108.2017.4(56).100-107}, \texttt{URL:} \url{http://stsam.irgups.ru/sites/default/files/articles\_pdf\_files/100-107.pdf}
\bibitem{Kopay} N. Kopaigorodsky. Use of ontologies in semantic information systems. // Journal "Ontology of Design", 4(14), 2014, p. 78--89 (in Russian) \texttt{URL:} \url{http://agora.guru.ru/scientific\_journal/files/Ontology\_Of\_Designing\_4\_2014\_opt1.pdf}
\bibitem{MDA} D. Frankel. Model Driven Architecture: Applying MDA to Enterprise Computing. Wiley; 1 edition, 2003, 352 p.
 \bibitem{Clio} J. Wielemaker, W. Beek, M. Hildebrand, J. Ossenbruggen. ClioPatria: A SWI-Prolog Infrastructure for the Semantic Web. Semantic Web, vol. 7, no. 5, 2016, pp. 529-541. \texttt{DOI:} \url{10.3233/SW-150191}

\bibitem{org} Org mode fantastic examples: \url{http://ehneilsen.net/notebook/orgExamples/org-examples.html}.
\bibitem{odmprof} ODM UML profile for OWL: \url{http://www.omg.org/spec/ODM/1.0/PDF/}.
\bibitem{odnex} OMG Ontology Domain Modeling example: \url{https://thematix.com/tools/vom/}.
\bibitem{odmvis} OWL UML Visualizer: \url{http://owlgred.lumii.lv/}.
\bibitem{uml2owd} UMLtoOWL: Converter from UML to OWL: \url{http://www.sfu.ca/~dgasevic/projects/UMLtoOWL/}. Uses XSLT to convert from XMI to OWL.
\bibitem{atmo3} AToM3: A tool for multi-paradigm modeling. \url{http://atom3.cs.mcgill.ca/}.
\bibitem{GT} Belghiat, Aissam \& Bourahla, Mustapha. (2012). UML Class Diagrams to OWL Ontologies: A Graph Transformation based Approach. International Journal of Computer Applications. 41. 41-46. 10.5120/5525-7566.
\bibitem{SWEB} \textbf{Semantic WEB Software Engineering}: \url{http://www.webist.org/Documents/Previous\_Invited\_Speakers/2012/WEBIST2012\_Pan.pdf}. Book: \url{https://www.iospress.nl/book/semantic-web-enabled-software-engineering/}.
\end{thebibliography}




% that's all folks
\end{document}

%%% Local Variables:
%%% End:
